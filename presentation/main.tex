\documentclass{beamer}

\usepackage[utf8]{inputenc}
\usepackage{epigraph}
\usepackage{tikz}
\usepackage{csquotes}

\usetikzlibrary{patterns, arrows, cd, positioning, backgrounds, fit, decorations.pathmorphing, shapes.misc, calc, matrix, decorations.pathreplacing}
\tikzset{%
  node/.style={draw, minimum size=12mm, inner sep=0mm, circle, fill=gray!50},
  edge/.style={thick},
  move/.style={->},
  brace/.style={
    decorate,
    decoration={brace,amplitude=10pt,mirror},
  },
  table nodes/.style={
    rectangle,
    draw=black,
    align=center,
    minimum height=7mm,
    text depth=0.5ex,
    text height=2ex,
    inner xsep=0pt,
    outer sep=0pt
  },
  table/.style={
    matrix of nodes,
    row sep=\pgflinewidth,
    column sep=\pgflinewidth,
    nodes={
      table nodes
    }
  }
}



\usetheme{metropolis}
\metroset{numbering=fraction}

\DeclareMathOperator{\din}{\delta_{\mathit{in}}}
\DeclareMathOperator{\dout}{\delta_{\mathit{out}}}
\DeclareMathOperator{\Tid}{T_{\text{id}}}
\DeclareMathOperator{\Tsteps}{T_{\text{steps}}}
\DeclareMathOperator{\Twalk}{T_{\text{walk}}}
\DeclareMathOperator{\Thead}{T_{\text{head}}}

\title{A Grimm idea}
\subtitle{Exploring graphs with pebbles}
\author{Christoph Welzel}
\institute{Logik und Theorie diskreter Systeme, RWTH Aachen}

\begin{document}
\maketitle
\begin{frame}
  \frametitle{Motivation}
  \begin{itemize}
    \item Traversing Brobdingnagian graphs
    \item[$\rightarrow$] Web crawlers
    \item Agent moves over vertices along edges
    \item[$\Rightarrow$] Memory efficient agents
  \end{itemize}
\end{frame}

\section{Graph traversial}
\begin{frame}
  \frametitle{Agents}
  \begin{quotation}
    Und als der volle Mond aufgestiegen war, so nahm Hänsel sein
    Schwesterchen an der Hand und ging den Kieselsteinen nach, die schimmerten
    wie neu geschlagene Batzen und zeigten ihnen den Weg.
  \end{quotation}
  \vspace{-0.5cm}
  \flushright{Brüder Grimm}
  \vspace{-0.5cm}
  \flushright{\emph{Hänsel und Gretel}}
  \begin{itemize}
    \item Agent carries set of markers $\rightarrow$ pebbles
    \item Leave pebbles on vertices
  \end{itemize}
\end{frame}

\begin{frame}
  \frametitle{Brobdignagian graphs}
  \begin{columns}
    \column{0.5\textwidth}
    \begin{itemize}
      \item Indistinguishable vertices
      \item Bounded degree ($\Delta$)
      \item Edges can locally enumerated $\left\{0,\dots,\Delta-1\right\}$
        (Port)
      \item Agent is aware of:
        \begin{enumerate}
          \item Degree  of vertex
          \item Pebbles placed on vertex
          \item Pebbles carried by agent
          \item Port    agent entered vertex
        \end{enumerate}
    \end{itemize}
    \column{0.5\textwidth}
    \begin{center}
      \resizebox{\textwidth}{!}{\begin{tikzpicture}
  \node[node] (m1) {$v_{1}$};
  \node[node,right=2cm of m1] (m2) {$v_{2}$};

  \node[above left=of m1] (e1) {};
  \node[below=of m1] (e2) {};

  \node[above right=of m2] (e3) {};
  \node[below=of m2] (e4) {};

  \draw[edge] (m1) to node[very near start, above] {0} (e1);
  \draw[edge] (m1) to node[near start, left] {2} (e2);

  \draw[edge] (m2) to node[very near start, above] {1} (e3);
  \draw[edge] (m2) to node[near start, right] {2} (e4);
  
  \draw[edge] (m1) to node[very near start, below] {1} node[very near end, above] {0} (m2);
\end{tikzpicture}
}
    \end{center}
  \end{columns}
\end{frame}

\begin{frame}
  \frametitle{Exploration sequence}
  \begin{columns}
    \column{0.5\textwidth}
    \begin{itemize}
      \item Describe movement of agent by relative turns
      \item $e_{1},\dots, e_{n}$
      \item For a vertex $v$ with degree $d_{v}$ which is entered on port $p$
        is left through port $p + e_{i}\mod p$
      \item<2->[$\rightarrow$] Example:
        $\alt<3-3>{\colorbox{green}{1}}{\colorbox{white!0}{1}},
        \alt<4-4>{\colorbox{green}{3}}{\colorbox{white!0}{3}},
        \alt<5-5>{\colorbox{green}{2}}{\colorbox{white!0}{2}},
        \alt<6-6>{\colorbox{green}{2}}{\colorbox{white!0}{2}}$
    \end{itemize}
    \column{0.5\textwidth}
    \begin{center}
      \resizebox{\textwidth}{!}{\begin{tikzpicture}
  \node[minimum size=12mm,inner sep=0mm, circle] (8) {};
  \node[node,right=of 8] (6) {};
  \node[node,right=of 6] (7) {};
  \node[node,below=of 8] (0) {};
  \node[node,below=of 6] (1) {};
  \node[node,below=of 7] (2) {};
  \node[node,below=of 0] (3) {};
  \node[node,below=of 1] (4) {};
  \node[node,below=of 2] (5) {};
  \node[node,above right=0.5cm and 1cm of 2] (8) {};
  \node[node,below right=0.5cm and 1cm of 2] (9) {};

  \draw[edge] (0) to node[very near start, above] {0} node[very near end, above] {1} (1);
  \draw[edge] (1) to node[very near start, below] {0} node[very near end, above] {1} (2);
  \draw[edge,bend right] (2.north west) to node[very near start, right] {0} node[very near end, above] {2} (0);
  \draw[edge] (0) to node[very near start, right] {1} node[very near end, left ] {0} (3);
  \draw[edge] (3) to node[very near start, above] {1} node[very near end, below] {1} (4);
  \draw[edge] (4) to node[very near start, below] {0} node[very near end, below] {1} (5);
  \draw[edge] (5) to node[very near start, above] {0} node[very near end, below] {2} (1);
  \draw[edge] (1) to node[very near start, left ] {3} node[very near end, left ] {2} (6);
  \draw[edge] (6) to node[very near start, above] {0} node[very near end, above] {1} (7);
  \draw[edge] (8) to node[very near start, above] {0} node[very near end, right] {0} (7);
  \draw[edge] (8) to node[very near start, above] {1} node[very near end, right] {2} (5);
  \draw[edge] (9) to node[very near start, below] {1} node[very near end, below] {3} (5);
  \draw[edge,bend right=20] (9) to node[very near start, below] {0} node[very near end, below] {1} (6);

  \begin{scope}[on background layer]%
      \node [fit=(0)] (layer) {};%
  \end{scope}
  \begin{scope}[on background layer]%
      \node [fit=(4)] (layer) {};%
  \end{scope}
  \only<3-3>{\begin{scope}[on background layer]%
      \node [fit=(0), fill=green!50] (layer) {};%
  \end{scope}}
  \only<4-4>{\begin{scope}[on background layer]%
      \node [fit=(3), fill=green!50] (layer) {};%
  \end{scope}}
  \only<5-5>{\begin{scope}[on background layer]%
      \node [fit=(4), fill=green!50] (layer) {};%
  \end{scope}}
  \only<6-6>{\begin{scope}[on background layer]%
      \node [fit=(3), fill=green!50] (layer) {};%
  \end{scope}}
  \only<7-7>{\begin{scope}[on background layer]%
      \node [fit=(4), fill=green!50] (layer) {};%
  \end{scope}}


\end{tikzpicture}
}
    \end{center}
  \end{columns}
\end{frame}

\begin{frame}
  \frametitle{Closed walk}
  \begin{theorem}
    A exploration sequence $(e_{1},\dots,e_{k})^{\ast}$ produces a closed walk.
  \end{theorem}
  \begin{itemize}
    \item Introduce triple $(v,l,i)$ with vertex $v$ is entered
      through port $l$ and $e_{i}$ is the next turn
    \item Note that there is exactly one $(v,l,i) \leadsto (v',l',i')$
    \item Furthermore assume $t_{1} = (v',l',i') \leadsto (v,l,i)$ and
      $t_{2} = (v'',l'',i'') \leadsto (v,l,i)$:
      \begin{itemize}
        \item $t_{1}$ and $t_{2}$ both entered $v$ through $l$: $v' = v''$
        \item $t_{1}$ and $t_{2}$ lead that $i$ is the next turn: $i' = i''$
        \item $l' + e_{i'} \mod d_{v'} = l'' + e_{i''} \mod d_{v''}$ and
          $l',l''\leq d_{v'} = d_{v''}$: $l' = l''$
      \end{itemize}
    \item Only finite triples, thus projection to vertices yields closed walk
  \end{itemize}
\end{frame}

\begin{frame}
  \frametitle{Exploring pebble machine}
  \alt<+>{
    \begin{theorem}[Reingold]
      There is an algorithm $A$ with $\mathcal{O}(n)$ space requirement
      producing a universal exploration sequence for every regular graph with
      $n$ vertices.
    \end{theorem}
  }{
    \begin{theorem}
      Thus, there is $(\mathcal{O}(1), 0, \mathcal{O}(\log n))$ that moves
      along a closed walk with $n$ distinct vertices (or explores the whole
      graph).
    \end{theorem}
  }
  \begin{columns}
    \column{0.5\textwidth}
    \begin{itemize}
      \item<3-> Establish 3-regularity
        \begin{itemize}
          \item Transform $v$ to $3\cdot d_{v}$ distinct vertices
          \item TODO: ADD BOUNDARY
        \end{itemize}
      \item<5-> Use $A$ and take first $k = 3n\cdot c$ ($c$ number of
        configurations $M_{A}$)
    \end{itemize}
    \column{0.5\textwidth}
    \uncover<4->{\resizebox{\textwidth}{!}{\begin{tikzpicture}
  \node[node] (m1) {};
  \node[node] (m2) [right = 4cm of m1] {};
  \node[node] (m3) [right = 4cm of m2] {};

  \draw[edge] (m1) to node[below,very near start] {0} node[above,very near end] {1} (m2);
  \draw[edge] (m2) to node[above,very near start] {0} node[below,very near end] {0} (m3);

  \begin{scope}[on background layer]
    \node[fit=(m1),fill=green!50] (cl1) {};
    \node[fit=(m2),fill=yellow!50] (cl2) {};
    \node[fit=(m3),fill=red!50] (cl3) {};
  \end{scope}

  \uncover<3->{
    \node[node] (r11) [below = 6cm of m1] {0};
    \node[node] (r12) [above = 2cm of r11] {1};
    \node[node] (r13) [below = 2cm of r11] {2};

    \node[node] (l11) [below = 6cm of m3] {0};
    \node[node] (l12) [above = 2cm of l11] {1};
    \node[node] (l13) [below = 2cm of l11] {2};

    \node       (anker) [below = 6.5cm of m2] {};
    \node[node] (mid0) [above right = 1cm and 1cm of anker] {0};
    \node[node] (mid1) [below right = 1cm and 1cm of anker] {1};
    \node[node] (mid2) [below = 3cm of anker] {2};
    \node[node] (mid3) [below left = 1cm and 1cm of anker] {3};
    \node[node] (mid4) [above left = 1cm and 1cm of anker] {4};
    \node[node] (mid5) [above = 3cm of anker] {5};

    \draw[edge] (r11) to node[left,near end] {1} node[right,near start] {0} (r12);
    \draw[edge] (r11) to node[left,near end] {0} node[right,near start] {1} (r13);
    \draw[edge,bend right] (r12.south west) to node[left,very near end] {1} node[left, very near start] {0} (r13.north west);

    \draw[edge] (l11) to node[left,near end] {1} node[right,near start] {0}(l12);
    \draw[edge] (l11) to node[left,near end] {0} node[right,near start] {1}(l13);
    \draw[edge,bend left] (l12.south east) to node[left,very near end] {1} node[left, very near start] {0} (l13.north east);

    \draw[edge] (mid0) to node[right,near start] {0} node[right,near end] {1} (mid1);
    \draw[edge] (mid1) to node[right,near start] {0} node[right,near end] {1} (mid2);
    \draw[edge] (mid2) to node[left,near start] {0} node[left,near end] {1} (mid3);
    \draw[edge] (mid3) to node[left,near start] {0} node[left,near end] {1} (mid4);
    \draw[edge] (mid4) to node[left,near start] {0} node[left,near end] {1} (mid5);
    \draw[edge] (mid5) to node[right,near start] {0} node[right,near end] {1} (mid0);

    \draw[edge] (mid0) to node[very near start, above] {2} node[very near end, above] {2} (l11);
    \draw[edge] (mid2) to node[very near start, above] {2} node[very near end, above] {2} (l13);
    \draw[edge,bend left=10] (mid4) to node[very near start, below] {2} node[very near end, above] {2} (l12);

    \draw[edge] (mid1) to node[very near start, above] {2} node[very near end, above] {2} (r11);
    \draw[edge] (mid3) to node[very near start, above] {2} node[very near end, above] {2} (r13);
    \draw[edge] (mid5) to node[very near start, below] {2} node[very near end, above] {2} (r12);

    \only<3>{
      \begin{scope}[on background layer]
        \node[fit=(r11) (r12) (r13),fill=green!50] (cl1b) {};
        \node[fit=(mid0) (mid1) (mid2) (mid3) (mid4) (mid5),fill=yellow!50] (cl2b) {};
        \node[fit=(l11) (l12) (l13),fill=red!50] (cl3b) {};
        \draw[draw=green!50] (cl1.south west) to (cl1b.north west);
        \draw[draw=green!50] (cl1.south east) to (cl1b.north east);
        \draw[draw=yellow!50] (cl2.south west) to (cl2b.north west);
        \draw[draw=yellow!50] (cl2.south east) to (cl2b.north east);
        \draw[draw=red!50] (cl3.south west) to (cl3b.north west);
        \draw[draw=red!50] (cl3.south east) to (cl3b.north east);
      \end{scope}
    }
  }

\end{tikzpicture}
}}
  \end{columns}
\end{frame}

\section{Simulation}
\begin{frame}
  \frametitle{Tape simulation via pebbles}
  \begin{itemize}
    \item Simulate tape of size $m$ by pebbles:
      \begin{itemize}
        \item<3-> Separate into memory blocks of size $m_{1}$
        \item<4-> Represent every block by one of $\left\{p_{0},\dots,p_{\frac{m}{m_{1}}-1}\right\}$ pebbles
      \end{itemize}
  \end{itemize}
  \uncover<2->{\begin{tikzpicture}
  \matrix (tape) [table, text width=7mm, name=tape]
    { 1 & 1 & 0 & 0 & 1 & 0 & 0 & 1 & 1 & 0 & 0 & 0 \\};

  \node[above=of tape-1-1.west] (s1) {};
  \node[below=of tape-1-1.west] (s2) {};

  \node[above=of tape-1-3.east] (s3) {};
  \node[below=of tape-1-3.east] (s4) {};

  \node[above=of tape-1-6.east] (s5) {};
  \node[below=of tape-1-6.east] (s6) {};

  \node[above=of tape-1-9.east] (s7) {};
  \node[below=of tape-1-9.east] (s8) {};

  \node[above=of tape-1-12.east] (s9) {};
  \node[below=of tape-1-12.east] (s10) {};

  \uncover<2-2>{\draw[brace] (tape-1-12.north east) -- node[yshift=6mm] {$m = 12$} (tape-1-1.north west);}
  \uncover<3->{\draw[brace] (s9.center) -- node[yshift=6mm] {$m = 12$} (s1.center);}

  \uncover<3->{
    \draw[dashed] (s1) to (s2);
    \draw[dashed] (s3) to (s4);
    \draw[dashed] (s5) to (s6);
    \draw[dashed] (s7) to (s8);
    \draw[dashed] (s9) to (s10);
  }

  \uncover<3-3>{
    \draw[brace] (tape-1-1.south west) -- node[yshift=-6mm] {$m_{1} = 3$} (tape-1-3.south east);

    \draw[brace] (tape-1-4.south west) -- node[yshift=-6mm] {$m_{1} = 3$} (tape-1-6.south east);

    \draw[brace] (tape-1-7.south west) -- node[yshift=-6mm] {$m_{1} = 3$} (tape-1-9.south east);

    \draw[brace] (tape-1-10.south west) -- node[yshift=-6mm] {$m_{1} = 3$} (tape-1-12.south east);
  }
  \uncover<4-4>{
    \draw[brace] (tape-1-1.south west) -- node[yshift=-6mm]  {$p_{0}$} (tape-1-3.south east);

    \draw[brace] (tape-1-4.south west) -- node[yshift=-6mm]  {$p_{1}$} (tape-1-6.south east);

    \draw[brace] (tape-1-7.south west) -- node[yshift=-6mm]  {$p_{2}$} (tape-1-9.south east);

    \draw[brace] (tape-1-10.south west) -- node[yshift=-6mm] {$p_{3}$} (tape-1-12.south east);
  }
\end{tikzpicture}
}
\end{frame}

\begin{frame}
  \frametitle{Tape simulation via pebbles}
  \begin{tikzpicture}
  \matrix (tape) [table, text width=7mm, name=tape]
    { 1 & 1 & 0 & 0 & 1 & 0 & 0 & 1 & 1 & 0 & 0 & 0 \\};
  \draw[brace] (tape-1-1.south west) -- node[yshift=-6mm] {$p_{0}$} (tape-1-3.south east); 
  \draw[brace] (tape-1-4.south west) -- node[yshift=-6mm] {$p_{1}$} (tape-1-6.south east); 
  \draw[brace] (tape-1-7.south west) -- node[yshift=-6mm] {$p_{2}$} (tape-1-9.south east); 
  \draw[brace] (tape-1-10.south west) -- node[yshift=-6mm] {$p_{3}$} (tape-1-12.south east); 
\end{tikzpicture}

  \begin{columns}
    \column{0.5\textwidth}
    \begin{itemize}
      \item Every pebbles needs $2^{m_{1}}$ distinct \enquote{states}
        \begin{itemize}
          \item Placement on a walk with at least $2^{m_{1}}$ distinct vertices
            yields \enquote{states}
        \end{itemize}
    \end{itemize}
    \column{0.5\textwidth}
    \resizebox{\textwidth}{!}{\begin{tikzpicture}
  \node[minimum size=12mm,inner sep=0mm, circle] (8) {};
  \node[node,right=of 8,label={100:$p_{0}$}] (6) {$6$};
  \node[node,right=of 6] (7) {$7$};
  \node[node,below=of 8,label={100:$p_{3}$}] (0) {$0$};
  \node[node,below=of 6] (1) {$1$};
  \node[node,below=of 7,label={$p_{1}$}] (2) {$2$};
  \node[node,below=of 0,label={45:$p_{2}$}] (3) {$3$};
  \node[node,below=of 1] (4) {$4$};
  \node[node,below=of 2] (5) {$5$};

  \draw[move] (0) to (1);
  \draw[move] (1) to (2);
  \draw[move,bend left] (2.south west) to (0);
  \draw[move] (0) to (3);
  \draw[move] (3) to (4);
  \draw[move] (4) to (5);
  \draw[move] (5) to (1);
  \draw[move] (1) to (6);
  \draw[move] (6) to (7);
  \draw[dashed, move, decorate,
  decoration={snake,amplitude=0.5cm, segment length=3cm}, bend right=55]
  (7.north west) to (0);
\end{tikzpicture}
}
  \end{columns}
\end{frame}

\begin{frame}
  \frametitle{Mechanisms of simulation}
  \begin{itemize}
    \item Simulation of $M_{\text{Walk}}$ which explores at least $2^{m_{1}}$
      distinct vertices
    \item Identify \emph{distinct} vertices of the walk
    \item Read from and write to simulated tape
    \item Preserve tape content through steps of simulated agent
  \end{itemize}
\end{frame}

\begin{frame}
  \frametitle{Simulation of $M_{\text{Walk}}$}
  \begin{itemize}
    \item $M_{\text{Walk}}$ can be implemented as
      $(\mathcal{O}(1), 0, \mathcal{O}(m_{1}))$ pebble machine
    \item Used variables/pebbles:
      \begin{itemize}
        \item $\Twalk$: stores $M_{\text{Walk}}$'s tape
        \item $\Tsteps$: stores number of taken steps
        \item $\Tid$: stores number of visited distinct vertices
        \item $p_{\text{start}}$: marks the beginning
      \end{itemize}
  \end{itemize}
\end{frame}

\begin{frame}
  \frametitle{Finding distinct vertices}
  Problem: Indistinguishable vertices $\leadsto$ next vertex \enquote{new}?
  \begin{columns}
    \column{0.5\textwidth}
    Solution:
    \begin{enumerate}
      \item Execute step
      \item Drop pebble $p_{\text{tmp}}$
      \item Store $\Twalk$ in $T'_{\text{walk}}$
      \item Restart $M_{\text{Walk}}$
      \item Execute steps until $p_{\text{tmp}}$ is observed
      \item If $\Twalk = T'_{\text{walk}}$ the vertex is new
      \item Otherwise start over
    \end{enumerate}
    \column{0.5\textwidth}
    \begin{tikzpicture}
  \node[minimum size=12mm,inner sep=0mm, circle] (8) {};
  \node[node,right=of 8,label={100:$p_{0}$}] (6) {\only<20-20>{$6$}};
  \node[node,right=of 6] (7) {};
  \node[node,below=of 8,label={100:$p_{3}$}] (0) {$0$};
  \node[node,below=of 6] (1) {$1$};
  \node[node,below=of 7,label={$p_{1}$}] (2) {$2$};
  \node[node,below=of 0,label={45:$p_{2}$}] (3) {$3$};
  \node[node,below=of 1] (4) {$4$};
  \node[node,below=of 2] (5) {$5$};
  \node[fill=blue!50, above right=-0.1cm and -0.1cm of 0] (pstart) {$p_{\text{start}}$};

  \draw[move] (0) to (1);
  \draw[move] (1) to (2);
  \draw[move,bend left] (2.south west) to (0);
  \draw[move] (0) to (3);
  \draw[move] (3) to (4);
  \draw[move] (4) to (5);
  \draw[move] (5) to (1);
  \draw[move] (1) to (6);
  \draw[move] (6) to (7);
  \draw[dashed, move, decorate,
  decoration={snake,amplitude=0.5cm, segment length=3cm}, bend right=55]
  (7.north west) to (0);

  % placement dummies:
  \begin{scope}[on background layer]
    \node[fit=(5)] (d1) {};
  \end{scope}
  \begin{scope}[on background layer]
    \node[fit=(4)] (d2) {};
  \end{scope}
  \begin{scope}[on background layer]
    \node[fit=(3)] (d3) {};
  \end{scope}
  \begin{scope}[on background layer]
    \node[fit=(2)] (d4) {};
  \end{scope}
  \begin{scope}[on background layer]
    \node[fit=(1)] (d5) {};
  \end{scope}
  \begin{scope}[on background layer]
    \node[fit=(0)] (d6) {};
  \end{scope}

  \only<2-2>{
    \begin{scope}[on background layer]
      \node[fit=(5), fill=green!50] (s0) {};
    \end{scope}
  }
  \only<3-4>{
    \begin{scope}[on background layer]
      \node[fit=(1), fill=green!50] (s1) {};
    \end{scope}
  }
  \only<4-6>{
    \node[fill=blue!50, above right=-0.1cm and -0.1cm of 1] (ptmp1) {$p_{\text{tmp}}$};
  }
  \only<5-5>{
    \begin{scope}[on background layer]
      \node[fit=(0), fill=red!50] (s2) {};
    \end{scope}
  }
  \only<6-6>{
    \begin{scope}[on background layer]
      \node[fit=(1), fill=red!50] (s3) {};
    \end{scope}
  }
  \only<7-7>{
    \begin{scope}[on background layer]
      \node[fit=(1), fill=green!50] (s4) {};
    \end{scope}
  }
  \only<8-9>{
    \begin{scope}[on background layer]
      \node[fit=(6), fill=green!50] (s5) {};
    \end{scope}
  }
  \only<9-18>{
    \node[fill=blue!50, above right=-0.1cm and -0.1cm of 6] (ptmp2) {$p_{\text{tmp}}$};
  }
  \only<10-10>{
    \begin{scope}[on background layer]
      \node[fit=(0), fill=red!50] (s6) {};
    \end{scope}
  }
  \only<11-11>{
    \begin{scope}[on background layer]
      \node[fit=(1), fill=red!50] (s7) {};
    \end{scope}
  }
  \only<12-12>{
    \begin{scope}[on background layer]
      \node[fit=(2), fill=red!50] (s8) {};
    \end{scope}
  }
  \only<13-13>{
    \begin{scope}[on background layer]
      \node[fit=(0), fill=red!50] (s9) {};
    \end{scope}
  }
  \only<14-14>{
    \begin{scope}[on background layer]
      \node[fit=(3), fill=red!50] (s10) {};
    \end{scope}
  }
  \only<15-15>{
    \begin{scope}[on background layer]
      \node[fit=(4), fill=red!50] (s12) {};
    \end{scope}
  }
  \only<16-16>{
    \begin{scope}[on background layer]
      \node[fit=(5), fill=red!50] (s13) {};
    \end{scope}
  }
  \only<17-17>{
    \begin{scope}[on background layer]
      \node[fit=(1), fill=red!50] (s13) {};
    \end{scope}
  }
  \only<18-18>{
    \begin{scope}[on background layer]
      \node[fit=(6), fill=red!50] (s14) {};
    \end{scope}
  }
  \only<19-20>{
    \begin{scope}[on background layer]
      \node[fit=(6), fill=green!50] (s14) {};
    \end{scope}
  }
\end{tikzpicture}

  \end{columns}
\end{frame}

\begin{frame}
  \frametitle{Memory management}
  \begin{columns}
    \column{0.5\textwidth}
    \begin{itemize}
      \item $\Thead$ stores head position
      \item $p_{\lfloor\frac{\Thead}{m_{1}}\rfloor}$: pebble of memory block
      \item $\mathit{off} = \Thead\mod m_{1}$: offset in memory block
      \item Reading:
        \begin{itemize}
          \item Retrieve encoding pebble
          \item Return bit with offset
        \end{itemize}
      \item Writing:
        \begin{itemize}
          \item Read bit
          \item On bit-flip move encoding pebble $2^{\mathit{off}}$ bits up or down
        \end{itemize}
    \end{itemize}
    \column{0.5\textwidth}
    \resizebox{\textwidth}{!}{\begin{tikzpicture}
  \matrix (tape) [table, text width=7mm, name=tape]
    { 1 & 1 & 0 & 0 & 1 & 0 & 0 & 1 & 1 & 0 & 0 & 0 \\};
  \draw[brace] (tape-1-1.south west) -- node[yshift=-6mm] {$p_{0}$} (tape-1-3.south east); 
  \draw[brace] (tape-1-4.south west) -- node[yshift=-6mm] {$p_{1}$} (tape-1-6.south east); 
  \draw[brace] (tape-1-7.south west) -- node[yshift=-6mm] {$p_{2}$} (tape-1-9.south east); 
  \draw[brace] (tape-1-10.south west) -- node[yshift=-6mm] {$p_{3}$} (tape-1-12.south east); 
\end{tikzpicture}
}
    \resizebox{\textwidth}{!}{\begin{tikzpicture}
  \node[minimum size=12mm,inner sep=0mm, circle] (8) {};
  \node[node,right=of 8,label={100:$p_{0}$}] (6) {$6$};
  \node[node,right=of 6] (7) {$7$};
  \node[node,below=of 8,label={100:$p_{3}$}] (0) {$0$};
  \node[node,below=of 6] (1) {$1$};
  \node[node,below=of 7,label={$p_{1}$}] (2) {$2$};
  \node[node,below=of 0,label={45:$p_{2}$}] (3) {$3$};
  \node[node,below=of 1] (4) {$4$};
  \node[node,below=of 2] (5) {$5$};

  \draw[move] (0) to (1);
  \draw[move] (1) to (2);
  \draw[move,bend left] (2.south west) to (0);
  \draw[move] (0) to (3);
  \draw[move] (3) to (4);
  \draw[move] (4) to (5);
  \draw[move] (5) to (1);
  \draw[move] (1) to (6);
  \draw[move] (6) to (7);
  \draw[dashed, move, decorate,
  decoration={snake,amplitude=0.5cm, segment length=3cm}, bend right=55]
  (7.north west) to (0);
\end{tikzpicture}
}
  \end{columns}
\end{frame}

\begin{frame}
  \frametitle{Memory management: Writing}
  \begin{itemize}
    \item Read bit
    \item On bit-flip: move pebble
  \end{itemize}
  \resizebox{\textwidth}{!}{\begin{tikzpicture}
  \node[minimum size=12mm,inner sep=0mm, circle] (8) {};
  \node[node,right=of 8,label={100:\alt<0-11>{$p_{0}$}{}}] (6) {$6$};
  \node[node,right=of 6] (7) {$7$};
  \node[node,below=of 8,label={100:$p_{3}$}] (0) {$0$};
  \node[node,below=of 6] (1) {$1$};
  \node[node,below=of 7,label={$p_{1}$}] (2) {$2$};
  \node[node,below=of 0,label={45:$p_{2}$}] (3) {$3$};
  \node[node,below=of 1,label={90:\alt<0-19>{}{$p_{0}$}}] (4) {$4$};
  \node[node,below=of 2] (5) {$5$};

  \draw[move] (0) to (1);
  \draw[move] (1) to (2);
  \draw[move,bend left] (2.south west) to (0);
  \draw[move] (0) to (3);
  \draw[move] (3) to (4);
  \draw[move] (4) to (5);
  \draw[move] (5) to (1);
  \draw[move] (1) to (6);
  \draw[move] (6) to (7);
  \draw[dashed, move, decorate,
  decoration={snake,amplitude=0.5cm, segment length=3cm}, bend right=55]
  (7.north west) to (0);

  \matrix (tape) [left=of 0, table, text width=7mm, name=tape]
    { 1 & \alt<0-12>{1}{0} & 0 & 0 & 1 & 0 & 0 & 1 & 1 & 0 & 0 & 0 \\};
  \draw[brace] (tape-1-1.south west) -- node[yshift=-6mm] {$p_{0}$} (tape-1-3.south east); 
  \draw[brace] (tape-1-4.south west) -- node[yshift=-6mm] {$p_{1}$} (tape-1-6.south east); 
  \draw[brace] (tape-1-7.south west) -- node[yshift=-6mm] {$p_{2}$} (tape-1-9.south east); 
  \draw[brace] (tape-1-10.south west) -- node[yshift=-6mm] {$p_{3}$} (tape-1-12.south east); 

  \begin{scope}[on background layer]
    \node[fit=(0)] (d0) {};
  \end{scope}
  \begin{scope}[on background layer]
    \node[fit=(1)] (d1) {};
  \end{scope}
  \begin{scope}[on background layer]
    \node[fit=(2)] (d2) {};
  \end{scope}
  \begin{scope}[on background layer]
    \node[fit=(3)] (d3) {};
  \end{scope}
  \begin{scope}[on background layer]
    \node[fit=(4)] (d4) {};
  \end{scope}
  \begin{scope}[on background layer]
    \node[fit=(5)] (d5) {};
  \end{scope}
  \begin{scope}[on background layer]
    \node[fit=(6)] (d6) {};
  \end{scope}
  \begin{scope}[on background layer]
    \node[fit=(7)] (d7) {};
  \end{scope}

  \uncover<2->{
    \node[above right=0.6cm and 0.0cm of tape-1-2] (marker) {\alt<2-12>{0}{}};
    \draw[red, thick, ->] (marker) to (tape-1-2.north);
  }
  \only<3-3>{
    \begin{scope}[on background layer]
      \node[fit=(0), fill=red!50] (s1) {};
    \end{scope}
  }
  \only<4-4>{
    \begin{scope}[on background layer]
      \node[fit=(1), fill=red!50] (s2) {};
    \end{scope}
  }
  \only<5-5>{
    \begin{scope}[on background layer]
      \node[fit=(2), fill=red!50] (s3) {};
    \end{scope}
  }
  \only<6-6>{
    \begin{scope}[on background layer]
      \node[fit=(0), fill=red!50] (s4) {};
    \end{scope}
  }
  \only<7-7>{
    \begin{scope}[on background layer]
      \node[fit=(3), fill=red!50] (s5) {};
    \end{scope}
  }
  \only<8-8>{
    \begin{scope}[on background layer]
      \node[fit=(4), fill=red!50] (s6) {};
    \end{scope}
  }
  \only<9-9>{
    \begin{scope}[on background layer]
      \node[fit=(5), fill=red!50] (s7) {};
    \end{scope}
  }
  \only<10-10>{
    \begin{scope}[on background layer]
      \node[fit=(1), fill=red!50] (s8) {};
    \end{scope}
  }
  \only<11-11>{
    \begin{scope}[on background layer]
      \node[fit=(6), fill=red!50] (s9) {};
    \end{scope}
  }
  \only<14-14>{
    \begin{scope}[on background layer]
      \node[fit=(0), fill=red!50] (s10) {};
    \end{scope}
  }
  \only<15-15>{
    \begin{scope}[on background layer]
      \node[fit=(1), fill=red!50] (s11) {};
    \end{scope}
  }
  \only<16-16>{
    \begin{scope}[on background layer]
      \node[fit=(2), fill=red!50] (s12) {};
    \end{scope}
  }
  \only<17-17>{
    \begin{scope}[on background layer]
      \node[fit=(0), fill=red!50] (s13) {};
    \end{scope}
  }
  \only<18-18>{
    \begin{scope}[on background layer]
      \node[fit=(3), fill=red!50] (s14) {};
    \end{scope}
  }
  \only<19-20>{
    \begin{scope}[on background layer]
      \node[fit=(4), fill=red!50] (s15) {};
    \end{scope}
  }

\end{tikzpicture}
}
\end{frame}

\begin{frame}
  \frametitle{Simulation}
  \begin{theorem}
    For every graph $G$ a $(s,p,2m)$ pebble machine $M$ can be simulated by a
    $(cs,p+c,m)$ pebble machine that simulates the walk of $M$ on $G$ or
    explores $G$.
  \end{theorem}
  \begin{itemize}
    \item There is a $(c,0,c\cdot\log n)$ pebble machine which explores
      a graph $G$ with $n$ vertices
    \item Applying above simulation $\log \log n$ times yields:
      \begin{theorem}
        For every Graph $G$ with $n$ vertices there is a
        $(c'^{\log\log n}\cdot c, \log\log n\cdot c', c)$ pebble machine
        which explores $G$.
      \end{theorem}
  \end{itemize}
\end{frame}

\section{Limitations of pebble machines}
\begin{frame}
  \frametitle{Limitations of pebbles}
  \begin{itemize}
    \item Construct graphs which cannot be traversed by agents with $p$ pebbles
    \item Such graphs are called $p$-barriers
      \begin{center}
        \resizebox{0.3\textwidth}{!}{\begin{tikzpicture}
\node[node] (v1) {$v_{1}$};
\node[node, below=of v1] (v2) {$v_{2}$};
\draw[edge] (v1) to (v2);

\node[node, right=of v1] (v3) {$v_{3}$};
\node[node, below=of v3] (v4) {$v_{4}$};
\draw[edge] (v3) to (v4);

\node[fit=(v1) (v2) (v3) (v4), draw=black] (b) {};
\draw[dashed, thick] (b.north) to (b.south);
\end{tikzpicture}
}
      \end{center}
    \item In the following: inductive construction of barriers
  \end{itemize}
\end{frame}

\begin{frame}
  \frametitle{Locality gadget}
  \begin{itemize}
    \item Replace edge by a $(p-1)$-barrier
    \item Enforce \enquote{locality} of pebbles
    \item An agent cannot traverse any gadget without \emph{all} his pebbles
    \item All pebbles of the agent are either carried or placed within the
      adjacent gadgets
  \end{itemize}
  \begin{center}
    \uncover<2->{\resizebox{0.8\textwidth}{!}{\begin{tikzpicture}
  \node[node] (0) {$v$};
  \node[node,above right=of 0] (1) {};
  \node[node,above left=of 0] (2) {};
  \node[node,below=of 0] (3) {};

  \draw[edge] (0) to (1);
  \draw[edge] (0) to (2);
  \draw[edge] (0) to (3);

  \node[node,right=5cm of 0] (0') {$v$};
  \node[node,above right=of 0'] (1') {};
  \node[node,above left=of 0'] (2') {};
  \node[node,below=of 0'] (3') {};

  \draw[edge,on background layer] (0') to node[draw,midway,fill=gray,shape=rectangle] {$B$} (1');
  \draw[edge,on background layer] (0') to node[draw,midway,fill=gray,shape=rectangle] {$B$} (2');
  \draw[edge,on background layer] (0') to node[draw,midway,fill=gray,shape=rectangle] {$B$} (3');

  \draw[->, shorten <=0.2cm, shorten >=0.2cm,
  decorate,decoration={snake,amplitude=0.3cm, segment length=1.0cm}] (0) to (0');
\end{tikzpicture}
}}
  \end{center}
\end{frame}

\begin{frame}
  \frametitle{Barrier construction: Initial thoughts}
  \begin{columns}
    \column{0.5\textwidth}
    \begin{itemize}
      \item Consider a subset of pebbles of cardinality $p$
      \item Use the locality gadget as edges
      \item Since pebbles are local: enumerate configurations in macro vertex
        $C_{1},\dots,C_{\alpha}$
      \item Project $A$ to a agent without pebbles $B$
      \item $B$ has $\alpha$ states and traverses through macro vertices
    \end{itemize}
    \column{0.5\textwidth}
    \resizebox{\textwidth}{!}{\begin{tikzpicture}
  \node[node] (a0) {$m_{0}$};
  \node[node,right=of a0] (a1) {$m_{1}$};

  \node[node,below=of a0] (b0) {$m_{0}$};
  \node[node,right=of b0] (b1) {$m_{1}$};

  \begin{scope}[on background layer]
    \node[fit=(a0)] (dummy1) {};
    \node[fit=(a1)] (dummy2) {};
    \node[fit=(b0)] (dummy3) {};
    \node[fit=(b1)] (dummy4) {};
  \end{scope}

  \node[rectangle,fill=gray!50, right=0.25 of a0] (b) {$B$};

  \draw[edge] (a0) to (b);
  \draw[edge] (b) to (a1);
  \draw[edge] (b0) to (b1);

  \only<1-1> {
    \begin{scope}[on background layer]
      \node[fit=(a0),fill=green!50] {};
      \node[fit=(b0),fill=blue!50] {};
    \end{scope}
  }
  \only<2-2> {
    \begin{scope}[on background layer]
      \node[fit=(b),fill=green!50] {};
      \node[fit=(b1),fill=blue!50] {};
    \end{scope}
  }
  \only<3-3> {
    \begin{scope}[on background layer]
      \node[fit=(a1),fill=green!50] {};
      \node[fit=(b1),fill=blue!50] {};
    \end{scope}
  }
\end{tikzpicture}
}
  \end{columns}
\end{frame}

\begin{frame}
  \frametitle{Barrier construction: The real deal}
  \begin{itemize}
    \item $B$ has no pebbles and is therefore prone to a $0$-barrier
    \item Apply locality gadget to all edges
    \item This yields a $p$-barrier for the chosen subset of pebbles
    \item Iterate $p$-barriers for all possible subsets
  \end{itemize}
  \begin{center}
    \resizebox{0.8\textwidth}{!}{\begin{tikzpicture}
  \node[node] (b0l0) {};
  \node[node,below=2 of b0l0] (b0l1) {};
  \node[node,right=of b0l0] (b0r0) {};
  \node[node,below=2 of b0r0] (b0r1) {};
  \node[fit=(b0l0) (b0l1) (b0r0) (b0r1),draw=black,label={90:$H_{1}$}] (b0) {};
  \draw[dashed, thick] (b0.north) to (b0.south);

  \node[node,right=of b0r0] (b1l0) {};
  \node[node,below=2 of b1l0] (b1l1) {};
  \node[node,right=of b1l0] (b1r0) {};
  \node[node,below=2 of b1r0] (b1r1) {};
  \node[fit=(b1l0) (b1l1) (b1r0) (b1r1),draw=black,label={90:$H_{2}$}] (b1) {};
  \draw[dashed, thick] (b1.north) to (b1.south);

  \node[minimum size=12mm,circle,right=of b1r0] (b2l0) {};
  \node[minimum size=12mm,circle,below=2 of b2l0] (b2l1) {};
  \node[minimum size=12mm,circle,right=of b2l0] (b2r0) {};
  \node[minimum size=12mm,circle,below=2 of b2r0] (b2r1) {};

  \node[node,right=of b2r0] (b3l0) {};
  \node[node,below=2 of b3l0] (b3l1) {};
  \node[node,right=of b3l0] (b3r0) {};
  \node[node,below=2 of b3r0] (b3r1) {};
  \node[fit=(b3l0) (b3l1) (b3r0) (b3r1),draw=black,label={90:$H_{{n}\choose{p}}$}] (b2) {};
  \draw[dashed, thick] (b1.north) to (b1.south);

  \draw[edge] (b0r0) to (b1l0);
  \draw[edge] (b0r1) to (b1l1);
  \draw[dashed] (b1r0) to (b2l0);
  \draw[dashed] (b1r1) to (b2l1);
  \draw[dashed] (b2r0) to (b3l0);
  \draw[dashed] (b2r1) to (b3l1);

  \draw[edge] (b0l0) to (b0l1);
  \draw[edge] (b3r0) to (b3r1);

  \node[fit=(b0) (b1) (b2), draw=black, label={90:$B$},inner sep=12mm] (b) {};
\end{tikzpicture}
}
  \end{center}
\end{frame}

\begin{frame}
  \begin{center}
    \Huge Thank you for your attention.
  \end{center}
\end{frame}

\end{document}
