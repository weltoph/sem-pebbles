\documentclass[oneside]{scrartcl}

\usepackage[utf8]{inputenc}

\usepackage{epigraph}
\usepackage[autostyle,german=guillemets]{csquotes}
\usepackage[ngerman, english]{babel}
\usepackage[style=numeric,backend=biber]{biblatex}
\usepackage{amsmath}
\usepackage{amssymb}
\usepackage{amsthm}
\usepackage{todonotes}
\usepackage{mdframed}
\usepackage{wrapfig}

\newtheorem{thm}{Theorem}
\newtheorem{definition}{Definition}
\newtheorem{lem}{Lemma}
\newtheorem{prop}{Proposition}

\addbibresource{bibliography.bib}

\DeclareMathOperator{\din}{\delta_{\mathit{in}}}
\DeclareMathOperator{\dout}{\delta_{\mathit{out}}}
\DeclareMathOperator{\Tid}{T_{\text{id}}}
\DeclareMathOperator{\Tsteps}{T_{\text{steps}}}
\DeclareMathOperator{\Twalk}{T_{\text{walk}}}
\DeclareMathOperator{\Thead}{T_{\text{head}}}

\usetikzlibrary{patterns, arrows, cd, positioning, backgrounds, fit, decorations.pathmorphing, shapes.misc, calc, matrix, decorations.pathreplacing}
\tikzset{%
  node/.style={draw, minimum size=12mm, inner sep=0mm, circle, fill=gray!50},
  edge/.style={thick},
  move/.style={->},
  brace/.style={
    decorate,
    decoration={brace,amplitude=10pt,mirror},
  },
  table nodes/.style={
    rectangle,
    draw=black,
    align=center,
    minimum height=7mm,
    text depth=0.5ex,
    text height=2ex,
    inner xsep=0pt,
    outer sep=0pt
  },
  table/.style={
    matrix of nodes,
    row sep=\pgflinewidth,
    column sep=\pgflinewidth,
    nodes={
      table nodes
    }
  }
}



\begin{document}


\setlength{\epigraphwidth}{0.7\textwidth}
\setquotestyle{german}

\title{Exploring graphs with pebbles}
\subtitle{A \emph{Grimm} idea}
\author{Christoph Welzel}
\maketitle

\epigraph{Und als der volle Mond aufgestiegen war, so nahm Hänsel sein %
Schwesterchen an der Hand und ging den Kieselsteinen nach, die schimmerten %
wie neu geschlagene Batzen und zeigten ihnen den Weg.}%
{\emph{Hänsel und Gretel}\\Brüder Grimm}

\setquotestyle{english}
\section{Introduction}
In this work we present the main results from \cite{pebbles}, i.e.
how to systematically traverse graphs of a very large size by agents.
Agents are initially located on a vertex of the graph and move along edges.
Because the memory of the agents is very limited we allow agents to use a set
of distinguishable tokens, called pebbles, which can be left on a vertex and
recollected by another visit of the same vertex. As we will see this is a very
beneficial extension since it is possible to obtain a very low memory footprint
for the agents while preserving the exploration range.

\section{Preliminaries}
For the further course we assume the following: Firstly, the considered
graphs are finite but very large and agents have limited memory. Therefore,
agents can not compute the identifier of vertices which renders all vertices
indistinguishable for these. Additionally, the degree of the vertices is
bounded by a constant $\Delta$. For every vertex $v$ we assume that the
connected edges can be locally enumerated with labels $0,\dots,d_{v} - 1$
where $d_{v}$ is the degree of $v$. These edge-labels are called \emph{ports}
and every edge has one port for each incident vertex. Note that no
relation between these ports is presumed.

\section{Graph Exploration}
We begin this section by formally defining the agents as
\emph{pebble machines}. This definition closely follows the notion of Turing
machines but adds mechanisms to model the movements of the agent as well as
the management of pebbles.
\begin{definition}[Pebble machine]
  An $(s,p,m)$-\emph{pebble machine} $\mathcal{M}$ is defined as a tuple
  $(Q,F,P,m,\delta,\din,\dout,q_{0})$ with
  \begin{enumerate}
    \item a finite set of states $Q$ with cardinality $s$
    \item a finite set of final states $F\subseteq Q$
    \item a finite set of distinguishable pebbles $P$ with cardinality $p$
    \item the length $m$ of the available memory tape
    \item a computation function $\delta$
    \item a function $\din$ that describes the movement onto a vertex
    \item a function $\dout$ that describes the leaving of a vertex
    \item an initial state $q_{0}$
  \end{enumerate}
  where the computation function $\delta$ models the behaviour of the pebble
  machine as
  \begin{equation*}
    \delta\colon (Q\setminus F)\times\left\{0,1\right\}\rightarrow
    Q\times\left\{0,1\right\}\times\left\{L,R\right\}
  \end{equation*}
  with the typical understanding of the computation of a Turing machine.
  Furthermore
  \begin{equation*}
    \din\colon Q\times 2^{P}\times 2^{P}\times
    \left\{0,\dots,\Delta - 1\right\}\times\left\{0,\dots,\Delta - 1\right\}
    \rightarrow Q
  \end{equation*}
  models how an agent moves upon a vertex $v$ where
  $\din(q,P_{1},P_{2},d,\ell) = q'$ means that $q$ is the current state of
  $\mathcal{M}, P_{1}$ is the set of pebbles it carries, $P_{2}$ the set
  of pebbles located on $v, d$ is the degree of $v$ and $\ell$ the port
  through which $\mathcal{M}$ enters $v$. Then, $\mathcal{M}$ starts its
  computation according to $\delta$ in state $q'$.  On the other hand $\dout$
  is defined as
  \begin{equation*}
    \dout\colon Q
    \rightarrow 2^{P}\times 2^{P}\times \left\{0,\dots,\Delta - 1\right\}
  \end{equation*}
  where $\dout(q) = (P_{1}, P_{2}, \ell)$ defines that $\mathcal{M}$ leaves
  a vertex $v$ through port $\ell$, drops the pebbles in $P_{1}\subseteq P$
  onto $v$ and carries the pebbles in $P_{2}\subseteq P$.

  We additionally demand consistency of pebbles, i.e. if the agent
  carried the pebbles in $P_{1}$ and found the pebbles in
  $P_{2}$, then $P_{1}$ and $P_{2}$ are disjoint.
  Furthermore, if the agent leaves carrying $P'_{1}$ and dropping $P'_{2}$
  those sets are disjoint and $P_{1}\cup P_{2} = P'_{1}\cup P'_{2}$.
\end{definition}

\begin{wrapfigure}{r}{0.4\textwidth}
  \begin{center}
    \resizebox{0.3\textwidth}{!}{\begin{tikzpicture}
  \node[node] (m) {};
  \node[above=of m] (d1) {};
  \node[below right=of m] (d2) {};
  \node[below left =of m] (d3) {};

  \node[right=1.5 of m] (as) {};

  \draw[dotted] (m) to (d1);
  \draw[dotted] (m) to (d2);
  \draw[dotted] (m) to (d3);

  \node[right=6 of m] (dm) {};
  \node[node,above=0.5 of dm] (m1) {};
  \node[node,below left =0.5 and 0.5 of dm] (m2) {};
  \node[node,below right=0.5 and 0.5 of dm] (m3) {};
  \node[above=of m1] (dd1) {};
  \node[below left =of m2] (dd2) {};
  \node[below right=of m3] (dd3) {};

  \node[left=1.5 of dm] (ae) {};

  \draw[decorate,decoration={snake, amplitude=0.2cm, segment length=1cm},->] (as) to (ae);

  \draw (m1) to (m2);
  \draw (m2) to (m3);
  \draw (m3) to (m1);
  \draw[dotted] (m1) to (dd1);
  \draw[dotted] (m2) to (dd2);
  \draw[dotted] (m3) to (dd3);
\end{tikzpicture}
}
  \end{center}
  \caption{Replace vertex by a 3-clique preserving 3-regularity.}
  \label{fig:blowup}
  \vspace{-0.5cm}
\end{wrapfigure}
In order to describe the movement of an agent we introduce the notion of an
exploration sequence $e = e_{0}, e_{1},\dots$ for agents which defines relative
movements.
Let $\ell = \ell_{0},\ell_{1},\dots$ be a sequence of variables.
$\ell_{i}$ denotes that the $i$-th visited vertex is entered through port
$\ell_{i}$. Then, an agent following $e$ leaves through port
$(\ell_{i} + e_{i}) \mod d_{v}$. We define $\ell_{0} = 0$ and use
$(e_{0},\dots,e_{k})^{\ast}$ for periodic exploration sequences, i.e. that
the $(k+1)$-th step is again $e_{0}$. We call an
exploration sequence universal for a class of graphs if an agent following
it explores any graph of that class.

Beginning from the following result of \citeauthor{logspacealg}
\cite{logspacealg} we present how we can obtain an
$(\mathcal{O}(1),0,\mathcal{O}(\log n))$-pebble machine that defines an agent
which performs a closed walk of at least $n$ distinct vertices in any graph
with bounded degree.
\begin{thm}[Reingold \cite{logspacealg}]
  \label{thm:logalg}
  There is an $\mathcal{O}(\log n)$-space algorithm producing a universal
  exploration sequence for any regular graph on $n$ vertices.
\end{thm}
First we show that an agent which follows a periodic exploration sequence
moves along a closed walk.
\begin{lem}[Closed Walk Lemma]
  An agent following an exploration sequence of the form
  $(e_{0},\dots,e_{k-1})^{\ast}$ in an undirected graph moves along a closed
  walk.
\end{lem}
\begin{proof}
  Consider a graph with vertices of the form $(v,\ell,i)$ and
  edges, given by the next movement of an agent following the exploration
  sequence if it entered $v$ through port $\ell$ and performs the $i$-th step
  of the exploration sequence. It is easy to see that for a vertex $(v,\ell,i)$
  the next vertex is unique.
  Furthermore we argue that there is at most one predecessor for every vertex.
  Examine two predecessor $(v',\ell',i')$ and $(v'',\ell'',i'')$ for
  $(v,\ell,i)$. Both enter $v$ by $\ell$, thus $v' = v''$. It is clear that
  $i'$ and $i''$ are equal because for both the next index considered index of
  the exploration sequence is $i$. Finally, $\ell' + e_{i'} \mod d_{v'} =
  \ell'' + e_{i''} \mod d_{v''}$ and since
  $v' = v''$, $e_{i'} = e_{i''}$ and $\ell',\ell'' < d_{v'} = d_{v''}$,
  $\ell' = \ell''$ follows.
  Because now every vertex has one outgoing and at most one incoming edge 
  a walk starting in $(v_{0},0,0)$ must eventually lead to $(v_{0},0,0)$ again,
  because there are only finitely many triples $(v,\ell,i)$ but the agent moves
  infinitely. Projecting this walk to the first component yields a closed walk
  on the original graph.
\end{proof}
In the following we show that the algorithm from Theorem \ref{thm:logalg}
can be adapted to output a periodic universal exploration sequence for every
graph with at most $n$ vertices.
\begin{lem}
  \label{lem:pul}
  There exists an $\mathcal{O}(\log n)$-space algorithm producing a universal
  exploration sequence $(e_{0},\dots,e_{k-1})^{\ast}$ for any 3-regular graph
  with at most $n$ vertices.
\end{lem}
\begin{wrapfigure}{r}{0.5\textwidth}
  \flushright{\resizebox{0.4\textwidth}{!}{\begin{tikzpicture}
  \node[node] (m1) {};
  \node[node] (m2) [right = 4cm of m1] {};
  \node[node] (m3) [right = 4cm of m2] {};

  \draw[edge] (m1) to node[below,very near start] {0} node[above,very near end] {1} (m2);
  \draw[edge] (m2) to node[above,very near start] {0} node[below,very near end] {0} (m3);

  \begin{scope}[on background layer]
    \node[fit=(m1),fill=green!50] (cl1) {};
    \node[fit=(m2),fill=yellow!50] (cl2) {};
    \node[fit=(m3),fill=red!50] (cl3) {};
  \end{scope}

  \node[node] (r11) [below = 6cm of m1] {$0$};
  \node[node] (r12) [above = 2cm of r11] {$1$};
  \node[node] (r13) [below = 2cm of r11] {$2$};

  \node[node] (l11) [below = 6cm of m3] {$0$};
  \node[node] (l12) [above = 2cm of l11] {$1$};
  \node[node] (l13) [below = 2cm of l11] {$2$};

  \node       (anker) [below = 6.5cm of m2] {};
  \node[node] (mid0) [above right = 1cm and 1cm of anker] {$0$};
  \node[node] (mid1) [below right = 1cm and 1cm of anker] {$1$};
  \node[node] (mid2) [below = 3cm of anker] {$2$};
  \node[node] (mid3) [below left = 1cm and 1cm of anker] {$3$};
  \node[node] (mid4) [above left = 1cm and 1cm of anker] {$4$};
  \node[node] (mid5) [above = 3cm of anker] {$5$};

  \draw[edge] (r11) to node[left,near end] {1} node[right,near start] {0} (r12);
  \draw[edge] (r11) to node[left,near end] {0} node[right,near start] {1} (r13);
  \draw[edge,bend right] (r12.south west) to node[left,very near end] {1} node[left, very near start] {0} (r13.north west);

  \draw[edge] (l11) to node[left,near end] {1} node[right,near start] {0}(l12);
  \draw[edge] (l11) to node[left,near end] {0} node[right,near start] {1}(l13);
  \draw[edge,bend left] (l12.south east) to node[left,very near end] {1} node[left, very near start] {0} (l13.north east);

  \draw[edge] (mid0) to node[right,near start] {0} node[right,near end] {1} (mid1);
  \draw[edge] (mid1) to node[right,near start] {0} node[right,near end] {1} (mid2);
  \draw[edge] (mid2) to node[left,near start] {0} node[left,near end] {1} (mid3);
  \draw[edge] (mid3) to node[left,near start] {0} node[left,near end] {1} (mid4);
  \draw[edge] (mid4) to node[left,near start] {0} node[left,near end] {1} (mid5);
  \draw[edge] (mid5) to node[right,near start] {0} node[right,near end] {1} (mid0);

  \draw[edge] (mid0) to node[very near start, above] {2} node[very near end, above] {2} (l11);
  \draw[edge] (mid2) to node[very near start, above] {2} node[very near end, above] {2} (l13);
  \draw[edge,bend left=10] (mid4) to node[very near start, below] {2} node[very near end, above] {2} (l12);

  \draw[edge] (mid1) to node[very near start, above] {2} node[very near end, above] {2} (r11);
  \draw[edge] (mid3) to node[very near start, above] {2} node[very near end, above] {2} (r13);
  \draw[edge] (mid5) to node[very near start, below] {2} node[very near end, above] {2} (r12);

  \begin{scope}[on background layer]
    \node[fit=(r11) (r12) (r13),fill=green!50] (cl1b) {};
    \node[fit=(mid0) (mid1) (mid2) (mid3) (mid4) (mid5),fill=yellow!50] (cl2b) {};
    \node[fit=(l11) (l12) (l13),fill=red!50] (cl3b) {};
    \draw[draw=green!50] (cl1.south west) to (cl1b.north west);
    \draw[draw=green!50] (cl1.south east) to (cl1b.north east);
    \draw[draw=yellow!50] (cl2.south west) to (cl2b.north west);
    \draw[draw=yellow!50] (cl2.south east) to (cl2b.north east);
    \draw[draw=red!50] (cl3.south west) to (cl3b.north west);
    \draw[draw=red!50] (cl3.south east) to (cl3b.north east);
  \end{scope}

\end{tikzpicture}
}}
  \caption{Illustration how to transform any graph to a 3-regular graph.}
  \label{fig:3reg}
  \vspace{-0.5cm}
\end{wrapfigure}
\begin{proof}
  Let $\mathcal{M}$ be the Turing machine(TM) of Theorem \ref{thm:logalg}.
  First we show that the exploration sequence produced by $\mathcal{M}$ is
  universal for every graph with at most $n$ vertices. Assume there
  is a 3-regular graph with $n_{0} < n$ vertices which the exploration
  sequence produced by $\mathcal{M}$ does not explore. Note that $n_{0}$ is
  even because every 3-regular graph has an even number of vertices by the
  Handshaking Lemma. Then, there is a vertex $v$ which is not explored. We can
  now repeatedly transform $v$ into a clique of three vertices with degree
  three (see Figure \ref{fig:blowup}).
  Thus, we added two vertices which are also not explored. This can be iterated
  until there are $n$ vertices. All the added vertices are not explored in a
  graph with $n$ vertices contradicting Theorem \ref{thm:logalg}.

  Secondly, we show that we can construct a TM $\mathcal{M}'$ which simulates
  $\mathcal{M}$ until it produces an exploration sequence of length
  $k = 3n\cdot c$ where $c$ is the number of configurations of $\mathcal{M}$.
  $\mathcal{M}'$ outputs now $(e_{0},\dots,e_{k-1})^{\ast}$ which is universal
  for any 3-regular graph with $n$ vertices. If we assume that it is not, we
  can conclude that we loop in a set of vertices $V'$ with $|V'|<n$. By a
  combinatorial argument we see that the produced exploration sequence $e$
  of $\mathcal{M}$ always loops in $V'$ because an agent following
  $e$ would enter a vertex twice by the same port whilst $\mathcal{M}$
  is in the same configuration both times.
\end{proof}
These lemmata are used in \cite{pebbles} to give the following theorem.
\begin{thm}
  \label{thm:pebblewalk}
  There exists a\\ $(\mathcal{O}(1),0,\mathcal{O}(\log n))$-pebble machine
  that moves along a closed walk and either explores the graph or visits
  at least $n$ distinct vertices, for any graph with bounded degree.
\end{thm}
\begin{proof}[Proof-Sketch]
  The TM constructed in Lemma \ref{lem:pul} is used as pebble
  machine after the graph is transformed to be 3-regular.
  We refer to Figure \ref{fig:3reg} which shows how every vertex $v$
  can be replaced by a gadget with $3\cdot d_{v}$ vertices of degree three. It
  is also noteworthy that this transformation can be done with
  $\mathcal{O}(\log n)$ memory: Because of the regular structure of the
  transformation (moving along port 0 or 1 stays in the gadget and only
  moving along port 2 changes the actual vertex). For details we refer to
  \cite{pebbles}.
\end{proof}

\section{Simulation}
In the following we present a crucial result from
\cite{pebbles} which is a simulation of a pebble machine which allows to store
the tape content of the simulated pebble machine as pebbles placed on a
walk within a given graph.
\begin{thm}
  \label{thm:simulation}
  There is a constant $c\in\mathbb{N}$, such that for any graph $G$ with
  bounded degree and any $(s,p,2m)$-pebble machine $\mathcal{M}$, there
  exists a $(cs,p+c,m)$-pebble machine $\mathcal{M}'$ that simulates the
  walk of $\mathcal{M}$ or explores $G$.
\end{thm}
\begin{proof}[Proof-Sketch]
  For a detailed description of mechanisms of the simulation we refer to
  \cite{pebbles}. At this point we present the conceptional ideas but encourage
  interested readers to look up the details.
  
  We separate the tape of the simulated machine into memory blocks of size
  $m_1$. With Theorem \ref{thm:pebblewalk} there is a
  $(\mathcal{O}(1),0,\mathcal{O}(m_{1}))$-pebble machine
  $\mathcal{M}_{\text{walk}}$ to perform a closed walk $\omega$ with
  $2^{m_{1}}$ distinct vertices. We use a pebble to mark the beginning vertex
  of this walk. Every memory block can now be represented by
  a pebble which encodes the block by its placement on $\omega$ (see
  Figure \ref{fig:tapesim}).

  Because the vertices are indistinguishable it is necessary to find a
  mechanism to enumerate distinct vertices in $\omega$ otherwise the state of
  pebbles might be ambiguous. This can be achieved by always marking the next
  vertex $v$ of $\omega$ with a pebble and check if $v$ is
  traversed by a new run of $\mathcal{M}_{\text{walk}}$ in fewer steps. If so,
  $v$ is not a new distinct vertex.

  The head position of the simulated machine is stored in a variable $\Thead$.
  In order to access this memory cell the state of the pebble with index
  $\left\lfloor \frac{\Thead}{m_{1}}\right\rfloor$ is extracted. Then the
  corresponding $(\Thead\mod m_{1})$-th bit is the bit under the current
  head-position of the simulated machine. If this bit is
  altered the corresponding pebble can be moved according to the new state of
  the memory block by $2^{(\Thead\mod m_{1})}$ indices on $\omega$.

  Finally, if the simulated machine executes a movement, it is necessary to
  carry the tape content along. Therefore the vertex $v'$ the machine moves
  onto is marked with a pebble and $\mathcal{M}_{\text{walk}}$ is used to find
  a new closed walk $\omega'$ starting in $v'$. Then all encoding pebbles are
  moved to the corresponding distinct vertex in $\omega'$ to preserve their
  states. (Note that $\mathcal{M}_{\text{walk}}$ gives a walk with at least
  $2^{m_{1}}$ distinct vertices for every starting vertex).
  \begin{figure}
    \begin{center}
      \resizebox{0.6\textwidth}{!}{\begin{tikzpicture}
  \node[minimum size=12mm,inner sep=0mm, circle] (8) {};
  \node[node,right=of 8,label={100:$p_{0}$}] (6) {$6$};
  \node[node,right=of 6] (7) {$7$};
  \node[node,below=of 8,label={100:$p_{3}$}] (0) {$0$};
  \node[node,below=of 6] (1) {$1$};
  \node[node,below=of 7,label={$p_{1}$}] (2) {$2$};
  \node[node,below=of 0,label={45:$p_{2}$}] (3) {$3$};
  \node[node,below=of 1] (4) {$4$};
  \node[node,below=of 2] (5) {$5$};

  \draw[move] (0) to (1);
  \draw[move] (1) to (2);
  \draw[move,bend left] (2.south west) to (0);
  \draw[move] (0) to (3);
  \draw[move] (3) to (4);
  \draw[move] (4) to (5);
  \draw[move] (5) to (1);
  \draw[move] (1) to (6);
  \draw[move] (6) to (7);
  \draw[dashed, move, decorate,
  decoration={snake,amplitude=0.5cm, segment length=3cm}, bend right=55]
  (7.north west) to (0);

  \matrix (tape) [left=of 0, table, text width=7mm, name=tape]
    { 1 & 1 & 0 & 0 & 1 & 0 & 0 & 1 & 1 & 0 & 0 & 0 \\};
  \draw[brace] (tape-1-1.south west) -- node[yshift=-6mm] {$p_{0}$} (tape-1-3.south east); 
  \draw[brace] (tape-1-4.south west) -- node[yshift=-6mm] {$p_{1}$} (tape-1-6.south east); 
  \draw[brace] (tape-1-7.south west) -- node[yshift=-6mm] {$p_{2}$} (tape-1-9.south east); 
  \draw[brace] (tape-1-10.south west) -- node[yshift=-6mm] {$p_{3}$} (tape-1-12.south east); 
\end{tikzpicture}
}
    \end{center}
    \caption{Illustration how to simulate the tape of a pebble machine by
      placing pebbles on a closed walk where $m_{1} = 3$. Note that the first
      $2^{m_{1}}=2^{3}=8$ distinct vertices are used for the simulation but the
    walk can be longer (as indicated by the dotted connection from $7$ to $0$).
  }
    \label{fig:tapesim}
  \end{figure}
  This concludes the intuition on how the tape of a pebble machine can be
  simulated by a set of pebbles. Each of these pebbles store a state by their
  placement in $\omega$ and can be managed to read and write from the tape.
  Furthermore the tape content can be preserved through movements of the
  simulated machine.
\end{proof}

As Theorem \ref{thm:pebblewalk} yields a $(c,0,c\cdot\log n)$ pebble machine
that explores a given graph we can now repeatedly apply the simulation
from Theorem \ref{thm:simulation} (precisely $\log\log n$ times) and obtain
a $(c'^{\log\log n}\cdot c, \log\log n\cdot c', c)$-pebble machine exploring
the graph as the following theorem from \cite{pebbles} states.
\begin{thm}
  Any bounded-degree graph of at most $n$ vertices can be explored using
  $\mathcal{O}(\log \log n)$ pebbles and memory.
\end{thm}

\section{Barriers}
In the following we lead through the discussion in \cite{pebbles} of the limits
of the exploration approach with pebbles. Specifically, we will argue that
there are 3-regular graphs that any agent with $p$ pebbles cannot traverse.
Such graphs are called $p$-\emph{barriers}.

The construction of a barrier is done iteratively and relies on a result from
\citeauthor{0barrier} \cite{0barrier} which is used to obtain an initial
$0$-barrier.
\begin{lem}
  \label{lem:0bar}
  For every $s$-state agent with $p$ pebbles $\mathcal{A}$ there exists a
  $0$-barrier with $\mathcal{O}(s^{2})$ vertices, which is independent of
  the starting state of $\mathcal{A}$.
\end{lem}

In the following we introduce a gadget construction which enforces
\enquote{locality} of pebbles. For an agent with $p$ pebbles we can exchange
every edge in a graph $G$ with a copy of a $(p-1)$-barrier $B$. We refer to the
original vertices as \emph{macro} vertices and to the constructed graph as
$G(B)$.  Since every edge is replaced by a $(p-1)$-barrier, an agent has to use
at least $p$ pebbles to cross this barrier which means that the agent keeps its
pebbles nearby because the agent cannot move along \enquote{edges} without
them. This leads to the following lemma.
\begin{lem}
  \label{lem:loc}
  Let $B$ be a $(p-1)$-barrier for an agent $\mathcal{A}$ with $p$ pebbles.
  Then, the following holds for any graph of the form $G(B)$.
  \begin{enumerate}
    \item $\mathcal{A}$ cannot get from a macro vertex $v$ to a distinct
      macro vertex $v'$ while using less than $p$ pebbles.
    \item At any time, there is some macro vertex $v$ such that $\mathcal{A}$
      and each pebble are at $v$ or in one of the surrounding gadgets $B(0),
      B(1), B(2)$
  \end{enumerate}
\end{lem}

With the help of this gadget construction we can now present how it is possible
to construct barriers iteratively.
\begin{thm}
  Given an $(r-1)$-barrier $B$ with $m$ vertices for an agent $\mathcal{A}$
  with $p\geq r$ pebbles, we can construct an $r$-barrier $B'$ with
  $\mathcal{O}({{p}\choose{r}}\cdot m\cdot \alpha_{B}^{2})$ vertices for
  $\mathcal{A}$.
\end{thm}
\begin{proof}
  For any subset of pebbles of cardinality $r$ we can use $B$ to apply the
  locality construction and obtain a graph $G(B)$. For $G(B)$ we can argue
  using Lemma \ref{lem:loc} that $\mathcal{A}$ has only a limited
  number of configurations within a macro vertex. Let $\alpha_{B}$ be the
  number of such configurations. Since the next macro vertex $\mathcal{A}$
  traverses is entirely dependent on the current configuration of $\mathcal{A}$
  we can project $\mathcal{A}$ to a pebble machine without pebbles
  $\mathcal{B}$ with $\alpha_{B}$ states which traverses macro vertices as
  $\mathcal{A}$ traverses macro vertices.
  Since $\mathcal{B}$ cannot use any pebbles we can
  obtain a $0$-barrier $C$ for it by Lemma \ref{lem:0bar}. Then, we can argue
  that $C(B)$ is the desired barrier for $\mathcal{A}$ which uses only the
  chosen subset of pebbles. In this manner a barrier for every ${p}\choose{r}$
  possible subsets of pebbles be obtained. Connecting these barriers gives
  the desired barrier $B'$. For the given bound of the vertices we refer to
  \cite{pebbles}.
\end{proof}
With a thorough combinatorial analysis of the amount of vertices for barriers
as it is performed in \cite{pebbles} the following theorem can be shown.
\begin{thm}
  For $r\leq p$ and $s \geq 2^{p}$, the number of vertices of the $r$-barrier
  $B_{r}$ for the $s$-state agent $\mathcal{A}$ with $p$ pebbles is bounded by
  $\mathcal{O}(s^{8^{r+1}})$.
\end{thm}
Using this theorem it is possible to show that any agent with sub-logarithmic
memory needs to use at least $\log\log n$ pebbles to explore a graph with $n$
vertices.
\begin{thm}
  For any constant $\epsilon < 0$, an agent with at most
  $\mathcal{O}((\log n)^{1-\epsilon})$ bits of memory needs at least
  $\Omega(\log\log n)$ distinguishable pebbles for exploring all graphs on at
  most $n$ vertices.
\end{thm}
For this result it is essential that from a $p$-barrier with $m$
vertices for an agent $\mathcal{A}$ with $p$ pebbles we can construct a
\emph{trap} for $\mathcal{A}$, i.e. a graph which $\mathcal{A}$ cannot
traverse entirely, with $2m+4$ vertices.

\section{Conclusion}
We presented the idea of pebble machines for graph exploration
as discussed in \cite{pebbles}. We focus mainly on the simulation of
memory of pebble machines. The idea to give pebbles a state by their placement
on a closed walk strikes as a very interesting approach. Furthermore, it is
also possible to find limitations of this approach in the form of barriers,
specifically that every agent with sub-logarithmic memory needs
$\Omega(\log\log n)$ pebbles to explore a graph with $n$ vertices. For agents
with $\mathcal{O}(\log\log n)$ memory $\Theta(\log\log n)$ pebbles are
necessary to explore graphs with $n$ vertices.

\setquotestyle{english}
\printbibliography
\end{document}
