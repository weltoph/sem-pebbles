\documentclass[draft,oneside]{scrartcl}

\usepackage[utf8]{inputenc}

\usepackage{epigraph}
\usepackage[autostyle,german=guillemets]{csquotes}
\usepackage[ngerman, english]{babel}
\usepackage[style=numeric,backend=biber]{biblatex}

\addbibresource{bibliography.bib}

\begin{document}


\setlength{\epigraphwidth}{0.7\textwidth}
\setquotestyle{german}

\title{A \emph{Grimm} idea}
\subtitle{Exploring graphs with pebbles}
\author{Christoph Welzel}
\maketitle

\begin{abstract}
  In this paper we re-present and explain an approach on exploring large
  undirected graphs with the help of markers, so called pebbles, for vertices.
  We focus on the central ideas of this approach and explain why it is possible
  to traverse any graph with $n$ vertices with $\Theta(\log \log n)$ pebbles
  and the same amount of memory.
\end{abstract}

\epigraph{\enquote{Aber, Bruder Wichsstiefel, wo hinaus geht der Weg?} %
  \enquote{Ich weiß es nicht}, antwortete der Jäger, %
  \enquote{ich habe mich in dem Wald verirrt.}}%
  {\emph{Die Stiefel von Büffelleder}\\Brüder Grimm}
\section{Introduction}
Traversing large graphs is an interesting challenge. In times of the internet
web crawlers have to traverse immense graphs but also regarding complexity
theory this is an interesting problem. First we will informally introduce two
complexity classes to illustrate the importance of graph traversal from a
complexity point of view: $L$ as the class of problems that can be solved by
a deterministic Turing machine with additional memory of logarithmic amount
of the input. And $SL$ as the class of problems that can be solved with the
same space requirements but by a non-deterministic symmetrical Turing machine,
where symmetrical means that for every computation step of the Turing machine
there is a computation step to undo the previous one. As it turns out checking
if two vertices $s$ and $t$ are connected in an undirected graph is a problem
that is complete for $SL$. This problem is often refered to as
$\mathrm{USTCON}$ and \citeauthor{logspacealg} showed in \cite{logspacealg}
that $\mathrm{USTCON}\in L$ which collapsed $SL$ into $L$, hence $SL = L$.
In the following we will present the results from \cite{pebbles} where another
mechanism of graph exploration is used: with the use of pebbles which are small
pieces of memory that can be dropped on and collected from the vertex the
agent currently traverses it is possible to traverse a graph with $n$ vertices
with $\mathcal{O}(\log\log n)$ memory and the same amount of pebbles.
Furthermore we will explain why there is no smaller amount of memory or pebbles
to generally obtain the same guarantee to traverse all $n$ vertices.


\epigraph{Und als der volle Mond aufgestiegen war, so nahm Hänsel sein %
Schwesterchen an der Hand und ging den Kieselsteinen nach, die schimmerten wie %
neu geschlagene Batzen und zeigten ihnen den Weg.}%
{\emph{Hänsel und Gretel}\\Brüder Grimm}
\section{Pebbel-Machines}

\setquotestyle{english}
\printbibliography
\end{document}
