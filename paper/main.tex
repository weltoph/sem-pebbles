theorem from \cite{pebbles} states.
\begin{thm}
  Any bounded-degree graph of at most $n$ vertices can be explored using
  $\mathcal{O}(\log \log n)$ pebbles and memory.
\end{thm}

\section{Barriers}
In the following we lead through the discussion in \cite{pebbles} of the limits
of the exploration approach with pebbles. Specifically, we will argue that
there are 3-regular graphs that any agent with $p$ pebbles cannot traverse.
Such graphs are called $p$-\emph{barriers}.

The construction of a barrier is done iteratively and relies on a result from
\citeauthor{0barrier} \cite{0barrier} which is used to obtain an initial
$0$-barrier.
\begin{lem}
  \label{lem:0bar}
  For every $s$-state agent with $p$ pebbles $\mathcal{A}$ there exists a
  $0$-barrier with $\mathcal{O}(s^{2})$ vertices, which is independent of
  the starting state of $\mathcal{A}$.
\end{lem}

In the following we introduce a gadget construction which enforces
\enquote{locality} of pebbles. For an agent with $p$ pebbles we can exchange
every edge in a graph $G$ with a copy of a $(p-1)$-barrier $B$. We refer to the
original vertices as \emph{macro} vertices and to the constructed graph as
$G(B)$.  Since every edge is replaced by a $(p-1)$-barrier, an agent has to use
at least $p$ pebbles to cross this barrier which means that the agent keeps its
pebbles nearby because the agent cannot move along \enquote{edges} without
them. This leads to the following lemma.
\begin{lem}
  \label{lem:loc}
  Let $B$ be a $(p-1)$-barrier for an agent $\mathcal{A}$ with $p$ pebbles.
  Then, the following holds for any graph of the form $G(B)$.
  \begin{enumerate}
    \item $\mathcal{A}$ cannot get from a macro vertex $v$ to a distinct
      macro vertex $v'$ while using less than $p$ pebbles.
    \item At any time, there is some macro vertex $v$ such that $\mathcal{A}$
      and each pebble are at $v$ or in one of the surrounding gadgets $B(0),
      B(1), B(2)$
  \end{enumerate}
\end{lem}

With the help of this gadget construction we can now present how it is possible
to construct barriers iteratively.
\begin{thm}
  Given an $(r-1)$-barrier $B$ with $m$ vertices for an agent $\mathcal{A}$
  with $p\geq r$ pebbles, we can construct an $r$-barrier $B'$ with
  $\mathcal{O}({{p}\choose{r}}\cdot m\cdot \alpha_{B}^{2})$ vertices for
  $\mathcal{A}$.
\end{thm}
\begin{proof}
  For any subset of pebbles of cardinality $r$ we can use $B$ to apply the
  locality construction and obtain a graph $G(B)$. For $G(B)$ we can argue
  using Lemma \ref{lem:loc} that $\mathcal{A}$ has only a limited
  number of configurations within a macro vertex. Let $\alpha_{B}$ be the
  number of such configurations. Since the next macro vertex $\mathcal{A}$
  traverses is entirely dependent on the current configuration of $\mathcal{A}$
  we can project $\mathcal{A}$ to a pebble machine without pebbles
  $\mathcal{B}$ with $\alpha_{B}$ states which traverses macro vertices as
  $\mathcal{A}$ traverses macro vertices.
  Since $\mathcal{B}$ cannot use any pebbles we can
  obtain a $0$-barrier $C$ for it by Lemma \ref{lem:0bar}. Then, we can argue
  that $C(B)$ is the desired barrier for $\mathcal{A}$ which uses only the
  chosen subset of pebbles. In this manner a barrier for every ${p}\choose{r}$
  possible subsets of pebbles be obtained. Connecting these barriers gives
  the desired barrier $B'$. For the given bound of the vertices we refer to
  \cite{pebbles}.
\end{proof}
With a thorough combinatorial analysis of the amount of vertices for barriers
as it is performed in \cite{pebbles} the following theorem can be shown.
\begin{thm}
  For $r\leq p$ and $s \geq 2^{p}$, the number of vertices of the $r$-barrier
  $B_{r}$ for the $s$-state agent $\mathcal{A}$ with $p$ pebbles is bounded by
  $\mathcal{O}(s^{8^{r+1}})$.
\end{thm}
Using this theorem it is possible to show that any agent with sub-logarithmic
memory needs to use at least $\log\log n$ pebbles to explore a graph with $n$
vertices.
\begin{thm}
  For any constant $\epsilon < 0$, an agent with at most
  $\mathcal{O}((\log n)^{1-\epsilon})$ bits of memory needs at least
  $\Omega(\log\log n)$ distinguishable pebbles for exploring all graphs on at
  most $n$ vertices.
\end{thm}
For this result it is essential that from a $p$-barrier with $m$
vertices for an agent $\mathcal{A}$ with $p$ pebbles we can construct a
\emph{trap} for $\mathcal{A}$, i.e. a graph which $\mathcal{A}$ cannot
traverse entirely, with $2m+4$ vertices.

\section{Conclusion}
We presented the idea of pebble machines for graph exploration
as discussed in \cite{pebbles}. We focus mainly on the simulation of
memory of pebble machines. The idea to give pebbles a state by their placement
on a closed walk strikes as a very interesting approach. Furthermore, it is
also possible to find limitations of this approach in the form of barriers,
specifically that every agent with sub-logarithmic memory needs
$\Omega(\log\log n)$ pebbles to explore a graph with $n$ vertices. For agents
with $\mathcal{O}(\log\log n)$ memory $\Theta(\log\log n)$ pebbles are
necessary to explore graphs with $n$ vertices.

\setquotestyle{english}
\printbibliography
\end{document}
