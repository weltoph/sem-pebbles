\documentclass[oneside]{scrartcl}

\usepackage[utf8]{inputenc}

\usepackage{epigraph}
\usepackage[autostyle,german=guillemets]{csquotes}
\usepackage[ngerman, english]{babel}
\usepackage[style=numeric,backend=biber]{biblatex}
\usepackage{amsmath}
\usepackage{amssymb}
\usepackage{amsthm}
\usepackage{todonotes}
\usepackage{mdframed}
\usepackage{wrapfig}

\newtheorem{thm}{Theorem}
\newtheorem{definition}{Definition}
\newtheorem{lem}{Lemma}
\newtheorem{prop}{Proposition}

\addbibresource{bibliography.bib}

\DeclareMathOperator{\din}{\delta_{\mathit{in}}}
\DeclareMathOperator{\dout}{\delta_{\mathit{out}}}
\DeclareMathOperator{\Tid}{T_{\text{id}}}
\DeclareMathOperator{\Tsteps}{T_{\text{steps}}}
\DeclareMathOperator{\Twalk}{T_{\text{walk}}}
\DeclareMathOperator{\Thead}{T_{\text{head}}}

\usetikzlibrary{patterns, arrows, cd, positioning, backgrounds, fit, decorations.pathmorphing, shapes.misc, calc, matrix, decorations.pathreplacing, external}
\tikzset{%
  node/.style={draw, minimum size=12mm, inner sep=0mm, circle, fill=gray!50},
  edge/.style={thick},
  move/.style={->},
  brace/.style={
    decorate,
    decoration={brace,amplitude=10pt,mirror},
  },
  table nodes/.style={
    rectangle,
    draw=black,
    align=center,
    minimum height=7mm,
    text depth=0.5ex,
    text height=2ex,
    inner xsep=0pt,
    outer sep=0pt,
    inner sep=0pt
  },
  table/.style={
    matrix of nodes,
    row sep=-0.5pt,
    column sep=-0.5pt,
    nodes={
      table nodes
    }
  }
}


\begin{document}


\setlength{\epigraphwidth}{0.7\textwidth}
\setquotestyle{german}

\title{Exploring graphs with pebbles}
\subtitle{A \emph{Grimm} idea}
\author{Christoph Welzel}
\maketitle

\epigraph{Und als der volle Mond aufgestiegen war, so nahm Hänsel sein %
Schwesterchen an der Hand und ging den Kieselsteinen nach, die schimmerten %
wie neu geschlagene Batzen und zeigten ihnen den Weg.}%
{\emph{Hänsel und Gretel}\\Brüder Grimm}

\setquotestyle{english}
\section{Introduction}
In the following we present the main results from \cite{pebbles}. These are
how to systematically traverse graphs of a very large size by agents. Such
agents are located on vertices of the graph and move along the edges. Because
the memory of the agents is very limited we allow agents to use a set of
distinguishable devices, called pebbles, which can be left on vertices and
regained by another visit on the same vertex. As we will see this is a very
rewarding extension since it is possible to obtain very low memory consumption
for the agents.

\section{Preliminaries}
For the further course we assume the following: Firstly, the considered
graphs are finite but huge. The considered agents have limited memory and can
therefore not compute the identifier of vertices. This makes all vertices
indistinguishable for the agents. Also, the degree of the vertices is bounded
by a constant $\Delta$. For every vertex $v$ we assume that the connected edges
can be locally enumerated with labels $0,\dots,d_{v} - 1$ where $d_{v}$ is the
degree of $v$. These edge-labels are called ports and every edge has one port
for both vertices it connects. Note that there is no relation between these
ports is presumed.

\section{Graph exploration}
We begin this section by formally defining our agents in the form of
\emph{pebble machines}. This definition closely follows the notion of Turing
machines but adds mechanisms to model the movements of the agent.
\begin{definition}[Pebble machine]
  A $(s,p,m)$-\emph{pebble machine} $\mathcal{M}$ is defined as a tuple
  $(Q,F,P,m,\delta,\din,\dout,q_{0})$. Here we have
  \begin{enumerate}
    \item a finite set of states $Q$ with cadinality $s$
    \item a finite set of final states $F\subseteq Q$
    \item a finite set of distinguishable pebbles $P$ with cardinality $p$
    \item the length of the available memory tape $m$
    \item a computation function $\delta$
    \item a function that describes the movement onto a vertex
      $\din$
    \item a function that describes the leaving of a vertex $\dout$
    \item an inital state $q_{0}$
  \end{enumerate}
  where formally the computation function models the behaviour of the pebble
  machine as
  \begin{equation*}
    \delta\colon (Q\setminus F)\times\left\{0,1\right\}\rightarrow
    Q\times\left\{0,1\right\}\times\left\{L,R\right\}
  \end{equation*}
  with the typical understanding of the computation of a Turing machine.
  Furthermore
  \begin{equation*}
    \din\colon Q\times 2^{P}\times 2^{P}\times
    \left\{0,\dots,\Delta - 1\right\}\times\left\{0,\dots,\Delta - 1\right\}
    \rightarrow Q
  \end{equation*}
  models how a pebble machine moves upon a vertex $v$ with
  $\din(q,P_{1},P_{2},d,\ell) = q'$ where $q$ is the current state of
  $\mathcal{M}$, $P_{1}$ the set of pebbles it carries, $P_{2}$ the set
  of pebbles located on $v$, $d$ is the degree of $v$ and $\ell$ the port
  through which $\mathcal{M}$ enters $v$. Then $\mathcal{M}$ starts its
  computation according to $\delta$ in state $q'$.

  $\dout$ is defined as
  \begin{equation*}
    \dout\colon Q
    \rightarrow 2^{P}\times 2^{P}\times \left\{0,\dots,\Delta - 1\right\}
  \end{equation*}
  where $\dout(q) = (P_{1}, P_{2}, \ell)$ defines that $\mathcal{M}$ leaves
  a vertex $v$ through port $\ell$ dropping the pebbles in $P_{1}$ onto $v$
  and carrying the pebble in $P_{2}$.

  We additionally demand consistency of pebbles, i.e. that if the agent
  carried the pebbles in $P_{1}\subseteq P$ and found the pebbles in
  $P_{2}\subseteq P$ it is necessary that $P_{1}$ and $P_{2}$ are disjoint and
  furthermore that if the agent leaves carrying $P'_{1}$ and leaving $P'_{2}$
  behind those sets are disjoint and $P_{1}\cup P_{2} = P'_{1}\cup P'_{2}$.
\end{definition}

In order to describe the movement of an agent we introduce the notion of an
exploration sequence as $e_{0}, e_{1},\dots$ which define relative movements:
Let $\ell = \ell_{0},\ell_{1},\dots$ be a sequence of variables where
$\ell_{i}$ denotes the port which a vertex $v$ was entered through. Then an
agent following $\ell$ leaves through port
$(\ell_{i} + e_{i}) \mod d_{v}$. We define $\ell_{0} = 0$ and the notion
$(e_{0},\dots,e_{k})^{\ast}$ for periodic exploration sequence.

Beginning from the following result of \cite{logspacealg} we present how we can
obtain an $(\mathcal{O}(1),0,\mathcal{O}(\log n))$ pebble machine that walks
along a closed walk of at least $n$ distinct vertices in any graph with bounded
degree.
\begin{thm}[Reingold]
  \label{thm:logalg}
  There is an $\mathcal{O}(\log n)$-space algorithm producing a universal
  exploration sequence for any regular graph on $n$ vertices.
\end{thm}
At first we show that an agent which follows a periodic exploration sequence
moves along a closed walk.
\begin{lem}[Closed Walk Lemma]
  An agent following an exploration sequence of the form
  $(e_{0},\dots,e_{k-1})^{\ast}$ in an undirected graph moves along a closed
  walk.
\end{lem}
\begin{proof}
  We therefore consider a graph with vertices of the form $(v,\ell,i)$ which
  denote that the agent entered in the original graph vertex $v$ through port
  $\ell$ and that the next element of the exploration sequence is $i$. For the
  edges we consider movements of the agent from the configuration given by the
  current vertex $(v,\ell,i)$. It is easy to see that for a vertex $(v,\ell,i)$
  the next vertex is deterministically chosen and is therefore unique. We
  furthermore argue that there is at most one predecessor for every vertex.
  Examine therefore two vertices $(v',\ell',i')$ and $(v'',\ell'',i'')$ which
  lead to $(v,\ell,i)$. Both enter $v$ by $\ell$ thus $v' = v''$. Furthermore
  it is clear that $i' + 1 \mod k = i = i'' + 1 \mod k$ which implies
  $i' = i''$. Concludingly,
  $\ell' + e_{i'} \mod d_{v'} = \ell'' + e_{i''} \mod d_{v''}$ and since
  $v' = v''$ and $e_{i'} = e_{i''}$ and $\ell',\ell'' < d_{v'} = d_{v''}$
  it follows that $\ell' = \ell''$.
  Because now every vertex has one outgoing and at most one incoming edge 
  a walk starting in $(v_{0},0,0)$ must loop
  because there are only finitly many triples $(v,\ell,i)$ but the agent moves
  infinitly. Projecting this walk to the first component yields a closed walk
  on the original graph.
\end{proof}
In the following we can show that the algorithm from Theorem \ref{thm:logalg}
can be adapted to output a periodic universal exploration sequence for every
graph with up to $n$ vertices.
\begin{lem}[Periodic Universal Lemma]
  \label{lem:pul}
  There exists an $\mathcal{O}(\log n)$-space algorithm producing a universal
  exploration sequence $(e_{0},\dots,e_{k-1})^{\ast}$ for any 3-regular graph
  with at most $n$ vertices.
\end{lem}
\begin{proof}
  Let $\mathcal{M}$ be the TM of Theorem \ref{thm:logalg}.
  At first we show that the exploration sequence produced by $\mathcal{M}$ is
  universal for every graph with at most $n$ vertices. Assume therefore there
  is a 3-regular graph with $n_{0} < n$ vertices for which $\mathcal{M}$
  produces a non universal exploration sequence. Note that $n_{0}$ is even
  because every 3-regular graph has an even amount of nodes by the Handshaking
  lemma. Then there is a vertex $v$ which is not explored. We can now
  repeatedly apply the following construction:
  \begin{center}
    \parbox{3\baselineskip}{\resizebox{!}{3\baselineskip}{\begin{tikzpicture}
  \node[node] (m) {};
  \node[above=of m] (d1) {};
  \node[below right=of m] (d2) {};
  \node[below left =of m] (d3) {};

  \node[right=1.5 of m] (as) {};

  \draw[dotted] (m) to (d1);
  \draw[dotted] (m) to (d2);
  \draw[dotted] (m) to (d3);

  \node[right=6 of m] (dm) {};
  \node[node,above=0.5 of dm] (m1) {};
  \node[node,below left =0.5 and 0.5 of dm] (m2) {};
  \node[node,below right=0.5 and 0.5 of dm] (m3) {};
  \node[above=of m1] (dd1) {};
  \node[below left =of m2] (dd2) {};
  \node[below right=of m3] (dd3) {};

  \node[left=1.5 of dm] (ae) {};

  \draw[decorate,decoration={snake, amplitude=0.2cm, segment length=1cm},->] (as) to (ae);

  \draw (m1) to (m2);
  \draw (m2) to (m3);
  \draw (m3) to (m1);
  \draw[dotted] (m1) to (dd1);
  \draw[dotted] (m2) to (dd2);
  \draw[dotted] (m3) to (dd3);
\end{tikzpicture}
}}
  \end{center}
  Thus, we added 2 vertices which are also not explored. This can be iterated
  until there are $n$ vertices for which all added vertices are not explored
  contradicting Theorem \ref{thm:logalg}.

  Secondly, we show that we can construct a TM $\mathcal{M}'$ which simulates
  $\mathcal{M}$ until it produced a exploration sequence of length $3n\cdot c$
  where $c$ is the number of configurations of $\mathcal{M}$. $\mathcal{M}'$
  outputs now $(e_{0},\dots,e_{k-1})^{\ast}$ which is universal for any
  3-regular graph with $n$ vertices. If we assume that it is not we can
  conclude that we loop in a set of vertices $V'$ for which $|V'|<n$. By a
  simple combinatorial argument we see that the produced exploration sequence
  of $\mathcal{M}$ therefore always loops in $V'$ because an agent following
  the produced exploration sequence would have entered a vertex twice by the
  same port whilst $\mathcal{M}$ was in the same configuration.
\end{proof}
\begin{wrapfigure}{r}{0.5\textwidth}
  \caption{Illustration how to transform any graph to a 3-regular graph.}
  \label{fig:3reg}
  \flushright{\resizebox{0.4\textwidth}{!}{\begin{tikzpicture}
  \node[node] (m1) {};
  \node[node] (m2) [right = 4cm of m1] {};
  \node[node] (m3) [right = 4cm of m2] {};

  \draw[edge] (m1) to node[below,very near start] {0} node[above,very near end] {1} (m2);
  \draw[edge] (m2) to node[above,very near start] {0} node[below,very near end] {0} (m3);

  \begin{scope}[on background layer]
    \node[fit=(m1),fill=green!50] (cl1) {};
    \node[fit=(m2),fill=yellow!50] (cl2) {};
    \node[fit=(m3),fill=red!50] (cl3) {};
  \end{scope}

  \node[node] (r11) [below = 6cm of m1] {$0$};
  \node[node] (r12) [above = 2cm of r11] {$1$};
  \node[node] (r13) [below = 2cm of r11] {$2$};

  \node[node] (l11) [below = 6cm of m3] {$0$};
  \node[node] (l12) [above = 2cm of l11] {$1$};
  \node[node] (l13) [below = 2cm of l11] {$2$};

  \node       (anker) [below = 6.5cm of m2] {};
  \node[node] (mid0) [above right = 1cm and 1cm of anker] {$0$};
  \node[node] (mid1) [below right = 1cm and 1cm of anker] {$1$};
  \node[node] (mid2) [below = 3cm of anker] {$2$};
  \node[node] (mid3) [below left = 1cm and 1cm of anker] {$3$};
  \node[node] (mid4) [above left = 1cm and 1cm of anker] {$4$};
  \node[node] (mid5) [above = 3cm of anker] {$5$};

  \draw[edge] (r11) to node[left,near end] {1} node[right,near start] {0} (r12);
  \draw[edge] (r11) to node[left,near end] {0} node[right,near start] {1} (r13);
  \draw[edge,bend right] (r12.south west) to node[left,very near end] {1} node[left, very near start] {0} (r13.north west);

  \draw[edge] (l11) to node[left,near end] {1} node[right,near start] {0}(l12);
  \draw[edge] (l11) to node[left,near end] {0} node[right,near start] {1}(l13);
  \draw[edge,bend left] (l12.south east) to node[left,very near end] {1} node[left, very near start] {0} (l13.north east);

  \draw[edge] (mid0) to node[right,near start] {0} node[right,near end] {1} (mid1);
  \draw[edge] (mid1) to node[right,near start] {0} node[right,near end] {1} (mid2);
  \draw[edge] (mid2) to node[left,near start] {0} node[left,near end] {1} (mid3);
  \draw[edge] (mid3) to node[left,near start] {0} node[left,near end] {1} (mid4);
  \draw[edge] (mid4) to node[left,near start] {0} node[left,near end] {1} (mid5);
  \draw[edge] (mid5) to node[right,near start] {0} node[right,near end] {1} (mid0);

  \draw[edge] (mid0) to node[very near start, above] {2} node[very near end, above] {2} (l11);
  \draw[edge] (mid2) to node[very near start, above] {2} node[very near end, above] {2} (l13);
  \draw[edge,bend left=10] (mid4) to node[very near start, below] {2} node[very near end, above] {2} (l12);

  \draw[edge] (mid1) to node[very near start, above] {2} node[very near end, above] {2} (r11);
  \draw[edge] (mid3) to node[very near start, above] {2} node[very near end, above] {2} (r13);
  \draw[edge] (mid5) to node[very near start, below] {2} node[very near end, above] {2} (r12);

  \begin{scope}[on background layer]
    \node[fit=(r11) (r12) (r13),fill=green!50] (cl1b) {};
    \node[fit=(mid0) (mid1) (mid2) (mid3) (mid4) (mid5),fill=yellow!50] (cl2b) {};
    \node[fit=(l11) (l12) (l13),fill=red!50] (cl3b) {};
    \draw[draw=green!50] (cl1.south west) to (cl1b.north west);
    \draw[draw=green!50] (cl1.south east) to (cl1b.north east);
    \draw[draw=yellow!50] (cl2.south west) to (cl2b.north west);
    \draw[draw=yellow!50] (cl2.south east) to (cl2b.north east);
    \draw[draw=red!50] (cl3.south west) to (cl3b.north west);
    \draw[draw=red!50] (cl3.south east) to (cl3b.north east);
  \end{scope}

\end{tikzpicture}
}}
\end{wrapfigure}
These lemmata are used in \cite{pebbles} to give the following theorem.
\begin{thm}
  \label{thm:pebblewalk}
  There exists a $(\mathcal{O}(1),0,\mathcal{O}(\log n))$-pebble machine
  that moves along a closed walk and either explores the graph or visits
  at least $n$ distinct vertices, for any graph with bounded degree.
\end{thm}
\begin{proof}[Sketch]
  The TM constructed in Lemma \ref{lem:pul} is used as the desired pebble
  machine. But first the graph has to be transformed to be 3-regular.
  We refer therefore to Figure \ref{fig:3reg} which shows how every vertex $v$
  can be replaced by $3\cdot d_{v}$ vertices which only have 3 edges. It is
  furthermore noteworthy that this transformation can be emulated with
  $\mathcal{O}(\log n)$ memory because of the regular structure of the
  transformation (moving along port 0 or 1 stays in the gadget and only
  moving along port 2 changes the actual vertex). For details we refer to
  \cite{pebbles}.
\end{proof}

\section{Simulation}
In the following we present possibly the most interesting result from
\cite{pebbles}. A simulation of a pebble machine which allows to store
the tape content of the simulated pebble machine as pebbles placed on a
defined walk within a graph.
\begin{thm}
  \label{thm:simulation}
  There is a constant $c\in\mathbb{N}$, such that for any graph $G$ with
  bounded degree and any $(s,p,2m)$-pebble machine $\mathcal{M}$, there
  exists a $(cs,p+c,m)$-pebble machine $\mathcal{M}'$ that simulates the
  walk of $\mathcal{M}$ or explores $G$.
\end{thm}
\begin{proof}
  In this proof we will focus on the simulation and we leave out a thorough
  analysis of the space requirements of the simulation since it is rather
  technical but does not add a lot to the idea. Nevertheless we encourage
  interested readers to see \cite{pebbles} for the details.

  The main idea of this simulation is to store the tape of $M$ by pebbles
  placed on a closed walk in the graph. We show that every node on this closed
  walk can be given an index and a pebble placed on a node represents the
  binary representation of the index as the content of $M$'s tape. Therefore we
  first off introduce a variable $m_{1}$ which is dependent on the simulated
  pebble machine $M$ (further requirements on $m_{1}$ and how to obtain it
  can be seen in \cite{pebbles}). We will now take the tape of $M$ and separate
  it into $2m/m_{1}$ blocks of the size $m_{1}$. And each of these blocks is
  encoded by the placement of a pebble on a closed walk $\omega$. So we use
  pebble $p_{0}$ to encode the first $m_{1}$ bits, the pebble $p_{1}$ used to
  encode the next $m_{1}$ bits and so forth. Thus, the pebble $p_{i}$ encodes
  the bits $im_{1},\dots,(i+1)m_{1}-1$. It is therefore necessary that every
  pebble can encode $2^{m_1}$ different states. For this we use a closed walk
  $\omega$ which visits at least $2^{m_1}$ distinct vertices yielding $2^{m_1}$
  many \enquote{states} for each pebble (by its placement on this closed walk).
  Therefore we can use the binary representation of the indices of the nodes
  on which the pebbles are placed as the content of the encoded memory block.
  Figure \ref{fig:tapesim} illustrates this.
  \begin{figure}[h!]
    \caption{Illustration how to simulate the tape of a pebble machine by
      placing pebbles on a closed walk where $m_{1} = 3$. Note that the first
      $2^{m_{1}}=2^{3}=8$ distinct vertices are used for the simulation but the
    walk can be longer (as indicated by the dotted connection from $7$ to $0$).}
    \label{fig:tapesim}
    \begin{center}
      \resizebox{0.6\textwidth}{!}{\input{tikz/tapesim}}
    \end{center}
  \end{figure}

  In the following we show how this simulation can be managed.
  Let $M_{\text{Walk}}$ be pebble machine that yields $\omega$ on $G$.
  $M_{\text{Walk}}$ can be implemented as $(\mathcal{O}(1),0,\mathcal{O}(m_{1}))$
  pebble machine by Theorem \ref{thm:pebblewalk}.
  In order to manage the simulation we use some
  variables: $\Tid, \Tsteps, \Twalk, \Thead$ where $\Tid$ is used to identify
  the index of a vertex on the walk of $M_{\text{Walk}}$, $\Twalk$ is the tape
  content of $M_{\text{Walk}}$ which can therefore be restarted by setting
  $\Twalk$ to 0, $\Tsteps$ counts the number of steps taken by
  $M_{\text{Walk}}$, and $\Thead$ stores where $M$'s head is positioned.
  Furthermore we need some pebbles to mark certain vertices of the closed walk:
  $p_{\text{start}}, p_{\text{tmp}}, p_{\text{next}}$ and additionally the
  pebbles that are placed on the walk to store $M$'s tape content
  $P_{M} = \left\{p_{0},p_{1},\dots,p_{2m/m_{1}-1}\right\}$.
  Now the different procedures used for the simulation can be presented:
  \begin{description}
    \item[Step():] Simulates on step according to $M_{\text{Walk}}$ and
      increases $\Tsteps$.
    \item[FindPebble($p_{i}$):] Executes \emph{Step()} until the pebble $p_{i}$
      is observed.
    \item[Restart():] Goes to the beginning of the closed walk by
      \emph{FindPebble($p_{\text{start}}$)} and sets $\Tsteps,\Tid,\Twalk$ to 0
    \item[NextDistinctVertex()] Finds the next distinct vertex of the walk. If
      there are no more vertices on the walk (which means that already
      $2^{m_1}$ distinct vertices have been traversed; the number of distinct
      vertices on the walk is given by $\Tid$) it simply executes
      \emph{Restart()}. Otherwise it does one step and drops $p_{\text{tmp}}$
      to mark this vertex. It then stores the current amount of steps as well
      as the current state of $M_{\text{Walk}}$ (introducing two auxiliary
      variables $\Tsteps',\Twalk'$). It then restarts $M_{\text{Walk}}$ and
      moves to $p_{\text{tmp}}$ by \emph{FindPebble($p_{\text{tmp}}$)}. If
      it used the same amount of steps as stored in $\Tsteps'$ then this is the
      first occurence of the vertex in the closed walk and therefore it is a
      new distinct vertex and $\Tid$ can be increased. Otherwise the context of
      $M_{\text{Walk}}$ is restored and the next vertex is checked.
    \item[GetPebbleId($p_{i}$):] Finds the index of the vertex $p_{i}$ is
      placed on. Beginning from the start of the closed walk
      (found by executing \emph{Restart()} it moves along distinct vertices
      (by using \emph{NextDistinctVertex()}) and returns $\Tid$ as soon as
      $p_{i}$ is observed.
    \item[PutPebbleAtId($p_{i},\mathit{id}$):] Puts pebble $p_i$ at a certain
      index. First it picks up $p_{i}$ and restarts the walk. Then it executes
      $\mathit{id}$ times \emph{NextDistinctVertex()} and drops $p_{i}$.
    \item[ReadBit():] Reads the $\Thead$'s bit of $M$'s tape. $\Thead$ gives
      the position of the bit that should be read. Therefore the pebble
      $p_{\lfloor \Thead/m_{1}\rfloor}$ encodes this bit. The \enquote{state}
      of this pebble can be obtained with
      \emph{GetPebbleId($p_{\lfloor \Thead/m_{1}\rfloor}$)} and from the
      encoded value the $(\Thead\mod m_{1})$-th bit is the one under $M$'s
      head.
    \item[WriteBit($b$):] Writes $b$ to $\Thead$'s bit of $M$'s tape. First
      $\Thead$'s bit is obtained (by \emph{ReadBit()}). Additionally we
      identify the encoding pebble $p_{i}$ as already seen for
      \emph{ReadBit()} and let be $j = \Thead\mod m_{1}$ the offset of the bit
      within the memory block. If the bit actually flips the value of the
      memory block changes by $2^j$ (for the flip from 1 to 0 the value
      decreases and for the flip from 0 to 1 it increases). Therefore $p_{i}$
      is put either $2^j$ indices up or down which can be easily done by using
      \emph{PutPebbleAtId(\dots)}.
  \end{description}
  These are the procedures used to simulate $M$'s tape. Furthermore we keep
  $M$'s states in the state space of the simulation. To conclude the simulation
  we have to consider what happens if $M$ moves from the current vertex to
  another vertex $v'$. Then the pebble $p_{tmp}$ is dropped on $v'$ and the
  pebbles are moved to the same index they have in $\omega$ for a new closed
  walk $\omega'$ (obtained by $M_{\text{Walk}}$ simulated starting in $v'$;
  this walk is also ensured to have at least $2^{m_1}$ distinct vertices).
\end{proof}

As Theorem \ref{thm:pebblewalk} yields a $(c,0,c\cdot\log n)$ pebble machine
that explores a given graph we can now repeatedly apply the simulation
from Theorem \ref{thm:simulation} (precisely $\log\log n$ times) and obtain
a $(c'^{\log\log n}\cdot c, \log\log n\cdot c', c)$-pebble machine exploring
the graph as the following theorem from \cite{pebbles} states.
\begin{thm}
  Any bounded-degree graph of at most $n$ vertices can be explored using
  $\mathcal{O}(\log \log n)$ pebbles and memory.
\end{thm}

\section{Barriers}
In the following we lead through the discussion in \cite{pebbles} of the limits
of the exploration approach with pebbles. Specifically, we will argue that
there are 3-regular graphs that any agent with $p$ pebbles cannot traverse.
Such graphs are called $p$-barriers.

This construction of barriers is iteratively and relies on a result from
\cite{0barrier} which is used to obtain an initial $0$-barrier.
\begin{lem}
  \label{lem:0bar}
  For every $s$-state agent with $p$ pebbles $\mathcal{A}$ there exists a
  $0$-barrier with $\mathcal{O}(s^{2})$ vertices, which is independent of
  the starting state of $\mathcal{A}$.
\end{lem}

In the following we introduce a gadget construction which enforces
\enquote{locality} of pebbles. For an agent with $p$ pebbles we can exchange
every edge in a graph $G$ with a copy of a $(p-1)$-barrier $B$. We refer to the
original vertices as \emph{macro} vertices and to the constructed graph as
$G(B)$.  Since every edge is replaced by a $(p-1)$-barrier an agent has to use
at least $p$ pebbles to cross this barrier which means that the agent keeps its
pebbles nearby because the agent cannot move along \enquote{edges} without
them.  This leads to the following lemma.
\begin{lem}
  \label{lem:loc}
  Let $B$ be a $(p-1)$-barrier for an agent $\mathcal{A}$ with $p$ pebbles.
  Then, the following holds for any graph of the form $G(B)$.
  \begin{enumerate}
    \item $\mathcal{A}$ cannot get from a macro vertex $v$ to a distinct
      macro vertex $v'$ while using less than $p$ pebbles.
    \item At any time, there is some macro vertex $v$ such that $\mathcal{A}$
      and each pebble are at $v$ or in one of the surrounding gadgets $B(0),
      B(1), B(2)$
  \end{enumerate}
\end{lem}

With the help of this gadget construction we can now present how it is possible
to construct barriers iteratively.
\begin{thm}
  Given an $(r-1)$-barrier $B$ with $m$ vertices for an agent $\mathcal{A}$
  with $p\geq r$ pebbles, we can construct an $r$-barrier $B'$ with
  $\mathcal{O}({{p}\choose{r}}\cdot m\cdot \alpha_{B}^{2})$ vertices for
  $\mathcal{A}$.
\end{thm}
\begin{proof}
  For any subset of pebbles of cardinality $p$ we can use $B$ to apply the
  locality construction and obtaining a graph $G(B)$. For $G(B)$ we can argue
  with the use of Lemma \ref{lem:loc} that $\mathcal{A}$ has only a limited
  amount of configurations within a macro vertex. Let $\alpha_{B}$ be the
  amount of these configurations. We can now project $\mathcal{A}$ to a pebble
  machine without pebbles $\mathcal{B}$ with $\alpha_{B}$ states which only
  traverses macro vertices. Since $\mathcal{B}$ cannot use any pebbles we can
  obtain by Lemma \ref{lem:0bar} a barrier $C$ for it. Then we can argue that
  $C(B)$ is the desired barrier for $\mathcal{A}$ which uses only the chosen
  subset of pebbles. For all ${p}\choose{r}$ possible subsets of pebbles a
  barrier can be obtained as described above. Connecting these barriers gives
  the desired barrier $B'$.
\end{proof}
For the given bound of the vertices we refer to \cite{pebbles}. The
analysis there contains a thorough combinatorial analysis of the amount of
vertices for barriers which we do not present here but which leads to the
following theorem.
\begin{thm}
  For $r\leq p$ and $s \geq 2^{p}$, the number of vertices of the $r$-barrier
  $B_{r}$ for the $s$-state agent $\mathcal{A}$ with $p$ pebbles is bounded by
  $\mathcal{O}(s^{8^{r+1}})$.
\end{thm}

Using this theorem it is possible to show that any agent with sublogarithmic
memory needs to use at least $\log\log n$ pebbles to explore a graph with $n$
vertices. For this result it is essential that from a $p$-barrier with $m$
vertices for an agent with $p$ pebbles we can construct a trap for this agent
with $2m+4$ vertices. Due to its technical nature we omit the proof to the
following theorem.
\begin{thm}
  For any constant $\epsilon < 0$, an agent with at most
  $\mathcal{O}((\log n)^{1-\epsilon})$ bits of memory needs at least
  $\Omega(\log\log n)$ distinguishable pebbles for exploring all graphs on at
  most $n$ vertices.
\end{thm}

\section{Conclusion}
In this paper re-presented the idea of pebble machines for graph exploration
as discussed in \cite{pebbles}. We lay the main focus on the simulation of
memory of pebble machines The idea to give pebbles a state by their placement
on a closed walk strikes as a very interesting approach. Furthermore it is also
possible to find limitations of this approach in the form of barriers.

\setquotestyle{english}
\printbibliography
\end{document}
